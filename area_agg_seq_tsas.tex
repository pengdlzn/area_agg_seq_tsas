\documentclass[acmlarge,review,natbib=false]{acmart}

%%drt24 hacks
%% Letter paper
%\setlength{\paperheight}{11in}
%\setlength{\paperwidth}{8.5in}

% Hack to try to make acmart work with biblatex: https://tex.stackexchange.com/questions/37076/is-it-possible-to-load-biblatex-with-a-class-that-has-already-loaded-natbib

\let\citename\relax
\RequirePackage[abbreviate=true, dateabbrev=true, isbn=true, doi=true, urldate=comp, url=true, maxbibnames=9, backref=false, backend=bibtex, style=ACM-Reference-Format, language=american,natbib]{biblatex}

%\usepackage{textcomp}
%\renewcommand{\textuparrow}{$\uparrow$}

%\addbibresource{sample-bibliography-biblatex.bib}
\addbibresource{Reference/BibReference.bib}

\renewcommand{\bibfont}{\Small}


\usepackage{booktabs} % For formal tables
% !TeX spellcheck = <none>
%%%%%%%%%%%%%%%%%%%%%%%%%%%%%%%%%%%%%%
%%%%%%%%%%%%%%%%%%%%%%%%%%%%%%%%%%%%%%
%FINAL VERSION:
%look for numbered but unlabelled equations
%\usepackage{refcheck}%compile twice in order to see the effects
%Acknowledgments

%update refereces doi url etc.





\usepackage[ruled]{algorithm2e} % For algorithms
\renewcommand{\algorithmcfname}{ALGORITHM}
\SetAlFnt{\small}
\SetAlCapFnt{\small}
\SetAlCapNameFnt{\small}
\SetAlCapHSkip{0pt}
\IncMargin{-\parindent}

% Metadata Information
\acmJournal{TSAS}
\acmVolume{0}
\acmNumber{0}
\acmArticle{00}
\acmYear{0000}
\acmMonth{0}
\acmArticleSeq{00}

%\acmBadgeR[http://ctuning.org/ae/ppopp2016.html]{ae-logo}
%\acmBadgeL[http://ctuning.org/ae/ppopp2016.html]{ae-logo}


% Copyright
%\setcopyright{acmcopyright}
%\setcopyright{acmlicensed}
%\setcopyright{rightsretained}
%\setcopyright{usgov}
\setcopyright{usgovmixed}
%\setcopyright{cagov}
%\setcopyright{cagovmixed}

% DOI
\acmDOI{0000001.0000001}

% Paper history
\received{February 2007}
\received{March 2009}
\received[accepted]{June 2009}

\usepackage{eurosym} %for the sign '€'
%\usepackage{etex}

%\usepackage[english]{babel}
%
\usepackage[utf8]{inputenc}
% choose options for [] as required from the list
% in the Reference Guide

%\usepackage{makeidx}         % allows index generation
\usepackage{graphicx}        % standard LaTeX graphics tool
% when including figure files
\usepackage{multicol}        % used for the two-column index
\usepackage[bottom]{footmisc}% places footnotes at page bottom
%\usepackage[disable]{todonotes}
%\usepackage{todonotes}
%\setlength{\marginparwidth}{3cm} %set the width of todonotes
\usepackage[defaultlines=4,all]{nowidow}%forbid widow or orphan
\usepackage{calc}

% Hack to try to make acmart work with biblatex: 
%https://tex.stackexchange.com/questions/37076/is-it-possible-to-load-biblatex-with-a-class-that-has-already-loaded-natbib
\let\citename\relax
\RequirePackage[
backend=biber,
style=numeric-comp, %number the references, compact
%style=authoryear, %no indices before names; year after names
%style=alphabetic, %style for my thesis
%
sorting=nyvt, %sort in bibliography by name, year, volume, title
sortcites, %sort in content by numbering
hyperref=true,%transform citations and back references into 
%clickable hyperlinks
backref=true, %to print back references in the bibliography
backrefstyle=three, %Compress three or more consecutive pages 
%to a range
%,
%citestyle=authoryear,
maxbibnames=9, %the author names appearing in bibliography
maxcitenames=2, %the author names appearing in text
giveninits=true,
%terseinits=true,%remove periods after initials of first names
isbn=false,
%doi=false,url=false,
eprint=false,
%dashed=false,
%useprefix=false,%sort bibliography use prefix like "van"
natbib=true,
style=ACM-Reference-Format,
language=american,
abbreviate=true, dateabbrev=true,
]{biblatex}

\addbibresource{Reference/BibReference.bib}


%if a reference has doi, then we print only doi in bibliography;
%otherwise, we print only url
%we don't use eprint
%we must deactivate eprint by using "eprint=false"
\renewbibmacro*{doi+eprint+url}{% 
	\iftoggle{bbx:url} 
	{\iffieldundef{doi}{\usebibmacro{url+urldate}}{}} 
	{}% 
	%	\newunit\newblock 
	%	\iftoggle{bbx:eprint} 
	%	{\usebibmacro{eprint}} 
	%	{}% 
	\newunit\newblock 
	\iftoggle{bbx:doi} 
	{\printfield{doi}} 
	{}}	





%remove the information we don't want to present
\AtEveryBibitem{% Clean up the bibtex rather than editing it
	\clearlist{address}
	\clearfield{date}
%	\clearfield{eprint}
	\clearfield{isbn}
	\clearfield{issn}
	\clearlist{location}
	\clearfield{month}
	\clearlist{language}
	%	\clearfield{series}
	
	\ifentrytype{book}{}{% Remove publisher and editor except for books
		\clearlist{publisher}
		%		\clearname{editor}
	}
}


\renewcommand*{\bibfont}{\small}%set the font size of references
%\renewcommand*{\nameyeardelim}{\addspace} %\remove the comma 
%%between author and year in citations
%\renewcommand*{\revsdnamepunct}{} %remove commas between last and first names in bibliography
%%\renewcommand{\labelnamepunct}{\addspace} %remove the punctuation after the year in bibliography
%\renewcommand{\finentrypunct}{} %remove the full stop at the end of each
%%bibliography entry
%%\fi
%
%%\iffalse
%We may have set prefix=false in \usepackage[]{biblatex}.
%Here, we turn on \useprefix in the actual document 
% i.e. in the citations and bibliography
\makeatletter
\AtBeginDocument{\toggletrue{blx@useprefix}}
\makeatother
%Since we want the prefix to be in lower-case in the bibliography, we issue
\renewbibmacro*{begentry}{\midsentence}
%%\fi

\usepackage[font=small,labelfont={bf,sf}]{caption}
\usepackage{subcaption}
\usepackage{bm} %Bold italic
\usepackage{gensymb}  % for degree sign, e.g., temperature
\usepackage{amsmath}  %align equations; text inside math
\usepackage{amstext}
\usepackage{amsfonts}  %for \mathbb{Z}
%\usepackage{amssymb} % for \smallsetminus
\usepackage{array}  %for define the fomats of columns in a table
\usepackage{dcolumn} %for \newcolumntype{d}[1]{D{'}{.}{#1}}
\usepackage[T1]{fontenc}
\usepackage{multirow}
\usepackage{makecell}
\usepackage{tabularx}
\usepackage{adjustbox}
\usepackage{capt-of}
\usepackage{subdepth} %fix the height problem of subscript
\usepackage{transparent}



%\newcommand{\revise}[2]{#2}
%\newcommand{\jan}[1]{{\color{red}#1}}
%\usepackage{siunitx} %for scientific 
%notation
%%we do this because of a bug of package floatrow
%\let\tmp\newinsert
%\let\newinsert\newbox
%\usepackage{floatrow}
%\let\newinsert\tmp


\usepackage{tikz}
\usepackage{pgfplots}
\usepackage{xspace}
\usepackage{booktabs} %for using \toprule, \midrule, and \bottomrule in a table



\graphicspath{{figures/}}

%%\iffalse
%For the references, the following commands make the titles 
%lowercase (sentence case),
%while keep book/journal/conference etc. names unchanged 
%\DeclareFieldFormat{sentencecase}{\MakeSentenceCase{#1}}
%\renewbibmacro*{title}{%
%	\ifthenelse{\iffieldundef{title}\AND\iffieldundef{subtitle}}
%	{}
%	{\ifthenelse{\ifentrytype{article}\OR\ifentrytype{inbook}%
%			\OR\ifentrytype{incollection}\OR\ifentrytype{inproceedings}%
%			\OR\ifentrytype{inreference}}
%		{\printtext[title]{%
%				\printfield[sentencecase]{title}%
%				\setunit{\subtitlepunct}%
%				\printfield[sentencecase]{subtitle}}}%
%		{\printtext[title]{%
%				\printfield[titlecase]{title}%
%				\setunit{\subtitlepunct}%
%				\printfield[titlecase]{subtitle}}}%
%		\newunit}%
%	\printfield{titleaddon}}
%
%%Remove the quotation marks for the titles in the reference
%\DeclareFieldFormat[article,inbook,incollection,inproceedings,
%patent,thesis,unpublished]{citetitle}{#1}
%\DeclareFieldFormat[article,inbook,incollection,inproceedings,
%patent,thesis,unpublished]{title}{#1}
%last name first in bibliography

\DeclareNameAlias{author}{family-given}%lastname before firstname
%%\fi



%%Using \DeclarePairedDelimiter from mathtools, you could define 
%%macros \ceil and \floor, which will scale the delimiters 
%%properly (if starred):
\usepackage{mathtools}
%\DeclarePairedDelimiter\ceil{\lceil}{\rceil}
%\DeclarePairedDelimiter\floor{\lfloor}{\rfloor}

\pgfplotsset{
	compat=newest,
	xlabel near ticks,
	ylabel near ticks
}

\newcommand{\Astar}{A$^{\!\star}$\xspace}
\newcommand{\AstarTwo}{A$^{\!\star}_{\,200{,}000}$\xspace}
\newcommand{\AstarFour}{A$^{\!\star}_{\,400{,}000}$\xspace}
\newcommand{\ILPOne}{ILP$_{100\,\mathrm{s}}$\xspace}
\newcommand{\ILPTwo}{ILP$_{200\,\mathrm{s}}$\xspace}

\newcommand{\tstar}{\ensuremath{t\xspace}}
\newcommand{\Pistar}{\ensuremath{\Pi}} % 
%\newcommand{\Comp}[1]{c_#1} % 
\newcommand{\comp}[1]{[#1]} % 
\newcommand{\edgenum}[1]{\|#1\|}
\newcommand{\area}[1]{\overline{#1}}
%{\Pi_{\mathrm{start}}(\tstar)}
\newcommand{\Pstart}{\ensuremath{P_\mathrm{start}}\xspace}
\newcommand{\Pgoal}{\ensuremath{P_\mathrm{goal}}\xspace}
\newcommand{\Pnode}{\ensuremath{P_{t,i}}\xspace}
\newcommand{\Psnode}{\ensuremath{P_{s,i}}\xspace}
\newcommand{\Tgoal}{\ensuremath{T_\mathrm{goal}}\xspace}




\newcommand{\e}[1]{\times 10^{#1}}
\newcommand{\fig}{Figure~}
\newcommand{\eq}{Equation~}
\newcommand{\fo}{Formula~}
\newcommand{\sect}{Section~}
\newcommand{\tab}{Table~}
\newcommand{\chap}{Chapter~}
\newcommand{\figs}{Figures~}
\newcommand{\eqs}{Equations~}
\newcommand{\fos}{Formulas~}
\newcommand{\sects}{Sections~}
\newcommand{\tabs}{Tables~}
\newcommand{\chaps}{Chapters~}
\newcommand{\myquad}[1][1]{\hspace*{#1em}\ignorespaces}
\newcommand{\eqquadVariable}{\myquad[6]}
\newcommand{\eqquadCost}{\myquad[2]}
\newcommand{\eqquadConstraintsX}{\myquad[4]}
\newcommand{\eqquadConstraintsYZ}{\myquad[3]}
\newcommand{\eqquadConstraintsC}{\myquad[3]}
\newcommand{\eqquadConstraints}{\myquad[3]}
\newcommand{\eqquadConstraintW}{\myquad[3]}
\newcommand{\inquad}{\qquad}
\newcommand{\inquadC}{\myquad[1]}

%because of \begin{array}
%\newcommand{\embshift}{\hspace{0.5pt}}%for tsas 
%\newcommand{\embshift}{\hspace{-3pt}}%for my thesis(10pt)
\newcommand{\embshift}{\hspace{-2.7pt}}%for my thesis(10.5pt)

%M or N are default tags. 
%The boxes using the same tag will have the same length. 
%The length is the maximum length of all the equations 
%in the boxes that use the same tag; see also 
%https://tex.stackexchange.com/questions/381502/how-to-align-text-in-multiple-separate-align-environments
\usepackage{eqparbox}
\newcommand{\embl}[2][M]
{\eqmakebox[#1][l]{$#2$}}
\newcommand{\embr}[2][N]
{\eqmakebox[#1][r]{$#2$}}
\newcommand{\embld}[2][M]
{\eqmakebox[#1][l]{$\displaystyle#2$}}
\newcommand{\embrd}[2][N]
{\eqmakebox[#1][r]{$\displaystyle#2$}}



\newcommand{\mypar}[1]{\bigskip\noindent\textbf{#1.}~}

%\renewcommand{\floatpagefraction}{.6}
%\renewcommand{\textfraction}{0.01}
%\renewcommand{\topfraction}{0.95}
%\renewcommand{\bottomfraction}{0.95}
%\renewcommand{\dbltopfraction}{0.95} % fit big float above 
%2-col. text
%\renewcommand\thesection{\arabic{section}}
%\renewcommand\thesubsection{\thesection.\arabic{subsection}}

% math-mode version of "l" column type
\newcolumntype{L}{>{$}l<{$}}
% math-mode version of "r" column type
\newcolumntype{R}{>{$}r<{$}}
% math-mode version of "c" column type
\newcolumntype{C}{>{$}c<{$}}
%align numbers by decimal points in table; 
%need package {dcolumn} and {array}
%see http://www.ctex.org/documents/packages/table/dcolumn.pdf 
\newcolumntype{d}[1]{D{'}{.}{#1}}
\newcolumntype{s}[1]{D{'}{~}{#1}}

%x:l, y:c, z:r, with specified width
%place text in a cell at top (b), middle (m), or bottom (p)
\newcolumntype{x}[1]
{>{\raggedright\arraybackslash\hspace{0pt}}m{#1}}
\newcolumntype{y}[1]
{>{\centering\arraybackslash\hspace{0pt}}m{#1}}
\newcolumntype{z}[1]
{>{\raggedleft\arraybackslash\hspace{0pt}}m{#1}}

%%math-mode version of "x,y,z" column type
%\newcolumntype{X}[1]
%{>{\raggedright\arraybackslash\hspace{0pt}}m{$#1$}}
%\newcolumntype{Y}[1]
%{>{\centering\arraybackslash\hspace{0pt}}m{$#1$}}
%\newcolumntype{Z}[1]
%{>{\raggedleft\arraybackslash\hspace{0pt}}m{$#1$}}




%\newcolumntype{y}[1]{>{\raggedleft\arraybackslash\hspace{0pt}}p{#1}}
%\newcolumntype{x}[1]{>{\centering\arraybackslash\hspace{0pt}}p{#1}}



%make package refcheck compatible with package hyperref 
\makeatletter
\AtBeginDocument{%
	\@ifpackageloaded{refcheck}{%
		\@ifundefined{hyperref}{}{%
			\let\T@ref@orig\T@ref
			\def\T@ref#1{\T@ref@orig{#1}\wrtusdrf{#1}}%
			\let\@refstar@orig\@refstar
			\def\@refstar#1{\@refstar@orig{#1}\wrtusdrf{#1}}%
			\DeclareRobustCommand\ref{\@ifstar\@refstar\T@ref}%
		}%
	}{}%
}
\makeatother





%%% Local Variables: 
%%% mode: latex
%%% TeX-master: "area_aggregation_seq_geoinformatica"
%%% End: 





%% Copyright
%%\setcopyright{none}
%%\setcopyright{acmcopyright}
%%\setcopyright{acmlicensed}
%\setcopyright{rightsretained}
%%\setcopyright{usgov}
%%\setcopyright{usgovmixed}
%%\setcopyright{cagov}
%%\setcopyright{cagovmixed}
%
%
%% DOI
%\acmDOI{10.475/123_4}
%
%% ISBN
%\acmISBN{123-4567-24-567/08/06}
%
%%Conference
%\acmConference[WOODSTOCK'97]{ACM Woodstock conference}{July 1997}{El
%  Paso, Texas USA}
%\acmYear{1997}
%\copyrightyear{2016}
%
%\acmPrice{15.00}

\usepackage[ruled]{algorithm2e} % For algorithms
\renewcommand{\algorithmcfname}{ALGORITHM}
\SetAlFnt{\small}
\SetAlCapFnt{\small}
\SetAlCapNameFnt{\small}
\SetAlCapHSkip{0pt}
\IncMargin{-\parindent}

% Metadata Information
\acmJournal{PACMHCI}
\acmVolume{9}
\acmNumber{4}
\acmArticle{39}
\acmYear{2010}
\acmMonth{3}
\acmArticleSeq{11}

%\acmBadgeR[http://ctuning.org/ae/ppopp2016.html]{ae-logo}
%\acmBadgeL[http://ctuning.org/ae/ppopp2016.html]{ae-logo}


% Copyright
%\setcopyright{acmcopyright}
%\setcopyright{acmlicensed}
%\setcopyright{rightsretained}
%\setcopyright{usgov}
\setcopyright{usgovmixed}
%\setcopyright{cagov}
%\setcopyright{cagovmixed}

% DOI
\acmDOI{0000001.0000001}

% Paper history
\received{February 2007}
\received{March 2009}
\received[accepted]{June 2009}


\begin{document}
\title{Finding Optimal Sequences for Area Aggregation  
	--\protect\\ 
	\Astar vs. Integer Linear Programming}
%\titlenote{Produces the permission block, and
%  copyright information}
%\subtitle{Extended Abstract}
%\subtitlenote{The full version of the author's guide is available as
%  \texttt{acmart.pdf} document}


\author{Dongliang Peng}
%\authornote{Dr.~Trovato insisted his name be first.}
\orcid{0000-0001-6848-3545}
\affiliation{%
  \institution{Chair of Computer Science I, 
  	University of W\"urzburg}
%  \streetaddress{P.O. Box 1212}
%  \city{Dublin}
%  \state{Ohio}
%  \postcode{43017-6221}
  \country{Germany}
}
\email{dongliang.peng@uni-wuerzburg.de}

\author{Alexander Wolff}
\orcid{0000-0001-5872-718X}
\affiliation{%
  \institution{Chair of Computer Science I, 
  	University of W\"urzburg}
  \country{Germany}
}
\email{www1.informatik.uni-wuerzburg.de/wolff}

\author{Jan-Henrik Haunert}
\orcid{0000-0001-8005-943X}
\affiliation{%
  \institution{Institute of Geodesy and Geoinformation,  
  University of Bonn}
  \country{Germany}
}
\email{www.geoinfo.uni-bonn.de/haunert}


%% The default list of authors is too long for headers}
%\renewcommand{\shortauthors}{B. Trovato et al.}


\begin{abstract}
To provide users with maps of different scales and 
to allow them to zoom in and out without losing context,
automatic methods for map generalization are needed.
Given two maps of the same region at two different scales,
a relevant problem is to compute maps for any intermediate scale.
We approach this problem for land-cover maps.
Given two land-cover maps at different scales, 
we want to find a sequence of small incremental
% To help map users not to lose context during zooming,
% we try to provide maps that change as continuous as possible.
% Since maps at some different scales often exist, 
% a relevant problem is to find maps at any intermediate scales.
% Given two land-cover maps of different scales, 
% we wish to find a sequence of small incremental
changes that gradually transforms one map into the other.
We assume that the two input maps consist of polygons 
each of which belongs to a given land-cover class. 
Every polygon on the smaller-scale map
is the union of a set of polygons on the larger-scale map. 

In each step of the sequence that we compute, 
the smallest area is merged with one of its neighbors. 
We do not select that neighbor according to a prescribed rule 
but compute the whole sequence of pairwise merges at once, 
based on global optimization.
We formalize this optimization problem as that of 
finding a shortest path in a (very large) graph.
For solving such a problem optimally, standard approaches
are the \Astar algorithm or integer linear programming.
We compare how the two methods solve our problem.
To avoid long computing times, we allow the two methods to 
return nonoptimal results.
In addition, we present a greedy algorithm 
as a benchmark.
We tested the three methods with a
data set of the official German topographic database ATKIS.
Our main result is that
\Astar finds optimal solutions for more instances 
than the other two methods
within a given time frame.
%
%
\end{abstract}

%
% The code below should be generated by the tool at
% http://dl.acm.org/ccs.cfm
% Please copy and paste the code instead of the example below.
%
\begin{CCSXML}
<ccs2012>
 <concept>
  <concept_id>10010520.10010553.10010562</concept_id>
  <concept_desc>Computer systems organization~Embedded systems</concept_desc>
  <concept_significance>500</concept_significance>
 </concept>
 <concept>
  <concept_id>10010520.10010575.10010755</concept_id>
  <concept_desc>Computer systems organization~Redundancy</concept_desc>
  <concept_significance>300</concept_significance>
 </concept>
 <concept>
  <concept_id>10010520.10010553.10010554</concept_id>
  <concept_desc>Computer systems organization~Robotics</concept_desc>
  <concept_significance>100</concept_significance>
 </concept>
 <concept>
  <concept_id>10003033.10003083.10003095</concept_id>
  <concept_desc>Networks~Network reliability</concept_desc>
  <concept_significance>100</concept_significance>
 </concept>
</ccs2012>
\end{CCSXML}

\ccsdesc[500]{Computer systems organization~Embedded systems}
\ccsdesc[300]{Computer systems organization~Redundancy}
\ccsdesc{Computer systems organization~Robotics}
\ccsdesc[100]{Networks~Network reliability}


\keywords{ACM proceedings, \LaTeX, text tagging}
\keywords{Continuous map generalization, Land cover, 
	 Type chang, Compactness, Graph, Optimization, Exponential}

\maketitle

\section{Introduction}
\label{sec:Introduction}

Maps are the main tool to represent geographical information. 
As geographical information is usually scale-dependent
\parencite{Muller1995Generalization,Weibel1997}, 
users need to have access to maps at different scales.
In our digital age, people realize this access by zooming on 
computers or mobile phones.
When users zoom in or out, 
a map must be changed to provide information 
appropriate to the corresponding zoom level.
A naive solution is 
to switch between maps at different scales.
This method, however, can result in 
large discrete changes, 
which tend to distract users.
A more advanced strategy is  
\emph{on-the-fly generalization}~\parencite{Weibel2017Fly},
which emphasizes on generalizing in real time.
Still, large discrete changes can be introduced.
To have a better zooming experience, 
users prefer the changes to be continuous.
This is why some digital maps such as Google Maps 
are trying to make zooming as continuous as they can,
though Google Maps is still based on switching between 
a certain number of maps at different scales.
The process of producing maps at any different scales
with smooth changes 
is called \emph{continuous map generalization} (CMG).
Ideally, there should be no discrete change in CMG.
However, the term is also used when 
the discrete changes are small enough not to be noticed as,
e.g., \textcite{Suba2016Road} state.

CMG has received a lot of attention
from cartographers and computer scientists.
\Textcite{vanKreveld2001} proposed five gradual changes
to support the continuous zooming of maps, 
which are \emph{moving}, \emph{rotating}, \emph{morphing}, 
\emph{fading}, and \emph{appearing}. 
He suggested using these gradual changes 
to adapt discrete generalization operators for CMG.
\textcite{Sester2005_CG} suggested simplifying building
footprints based on small incremental steps and 
to animate each step smoothly.
\textcite{Li2012Continuous} built hierarchies of road segments,
which they then used to omit road segments 
from lower levels of the hierarchy.
Moreover, they evaluated similarities 
between their results and existing maps.
\textcite{Peng2017Building} gradually grew
buildings to built-up areas by aggregating buildings 
whenever they become too close.
\textcite{Touya2017Progressive} progressively replaced
buildings with blocks. 
In addition, their method automatically inferred landmarks 
and put the landmarks on top of the blocks.
\textcite{Suba2016Road} continuously generalized road networks
which are represented as a set of areas.
Their method repeatedly finds the least-important area, 
and then either merges it with an adjacent area 
or collapses it to a line segment.
\textcite{Danciger2009} investigated the growing of regions, 
while preserving topology, area ratios, and
relative positions.
The strategy of using two maps at different scales
to generate intermediate-scale maps has been studied in multiple
representations, e.g., with respect to the selection of roads or
rivers~\parencite{Peng2012River,Girres2014}. 
Actually, this strategy is the key idea of the
morphing-based methods for CMG.
In order to morph from one polyline to another polyline,
which respectively represent, say, roads on a larger-scale map 
and a smaller-scale map, we first need to compute 
corresponding points between them (e.g., \citealp{Cecconi2003,Noellenburg2008,Chazal2010BallMap,Deng2015,Li2017_Building,Li2017Annealing}).
Then morphing can be realized by interpolating a set of 
intermediate polylines.
% Among the methods of computing corresponding points,
\textcite{Noellenburg2008} presented a method for computing an optimum
correspondence between two given polylines according to some cost function.
While straight-line trajectories 
are often used for interpolation
(e.g., \citealp{Cecconi2003,Deng2015}),
\textcite{Whited2011BallMorph} considered four other alternatives, i.e., \emph{hat}, \emph{tangent}, 
\emph{circular}, and \emph{parabolic} paths
based on so-called \emph{ball-map}
\parencite{Chazal2010BallMap}.
\textcite{Peng2013LSA} defined trajectories requiring that
angles in vertices and edge lengths should change linearly.
As these requirements may not agree with each other,
their method mediates between them using 
least-squares adjustments.
\textcite{Peng2016Admin} morphed county boundaries
to provincial boundaries.
For county boundaries that do not have corresponding 
provincial boundaries, they generated the correspondences based 
on \emph{compatible triangulations}.
\Textcite{vanOosterom2014tGAP} used
a data structure called
\emph{smooth topological generalized area partitioning}
to support visualizing CMG.
One of their contributions is that
a polygon is merged into another polygon continuously
by moving the boundary of the former.
\textcite{Huang2017Matrix} proposed a matrix-based structure 
to support CMG,
using a river network as an example.
For a given scale, 
their structure yields the rivers that should be kept 
as well as how much these rivers should be simplified.



\mypar{Optimization in map generalization}
Map generalization generally specifies 
and takes into account requirements
in order to produce maps of high quality
\parencite{Stoter2009Requirements}.
We categorize requirements as hard and soft constraints.
For example, when users zoom out, 
some land cover areas become too small to be seen.
These areas have to be aggregated.
When we aggregate one area into another, 
the type of the former is changed to the type of the latter. 
In this problem a hard constraint could be that 
we aggregate only two polygons at each step 
in order to keep changes small
(see for example \fig\ref{fig:AreaAgg_example}). 
A soft constraint could be that 
we wish to minimize the changes of types, e.g., 
we prefer aggregating a grass area into a farm area 
rather than into a settlement area.
This is a typical \emph{optimization} problem,
where we stick to hard constraints and 
try to fulfill soft ones as good as possible.
Optimization for map generalization is important 
not only because it finds optimal solutions,
but also because it helps us to evaluate the quality of a model
\parencite{Haunert2017Label,Haunert2008Assuring,Haunert2016Optimization}.
Recall that we wish to minimize the changes of types
when aggregating areas one by one.
A model could be to minimize 
the \emph{smallest} change over all the steps.
Using optimization, we are able to find optimal solutions
of this model at least for small instances.
If even the optimal solutions are bad,
then we can say that the model is not reasonable.
In this case, we should try another model, say, 
to minimize the \emph{average} change over all the steps.
Moreover, optimization is useful for evaluating heuristics.
We need heuristics because
many optimization problems cannot be solved efficiently
(e.g., \citealp{HaunertWolff2010AreaAgg,Haunert2016Partition}).
While heuristics can find some solutions in reasonable time,
it is important to know the quality of 
these solutions.
Fortunately, we can often find an optimal solution when 
the size of an instance is sufficiently small.
Consequently, we are able to evaluate 
the quality of a heuristic 
by comparing its results with optimal solutions
on small instances.


Optimization has been widely used in map generalization.
For example, \textcite{Harrie1999} displaced objects 
based on least-squares adjustments (LSA)
to solve spatial conflicts.
In his problem, the soft constraints 
for shapes and locations may contradict each other.
Therefore, it is necessary 
to mediate between these constraints, 
which is done by LSA.
\textcite{Sester2005Optimization} used LSA not only for 
displacing objects, but also for simplifying buildings. 
The output boundaries are required to 
be as close as possible to the original buildings.
\textcite{Tong2015AreaLSA} generalized land-cover areas,
where LSA was used to preserve the sizes of 
the land-cover areas.
\textcite{Regnauld2001} grouped buildings based on 
minimum spanning trees in order to typify
the buildings in a group.
\textcite{Burghardt2005Snakes} smoothed lines based on 
energy minimization. 
According to his setting, a line contains less energy
if it is smooth and close to the original line.
He repeatedly displace the line 
until finding a stable state 
in terms of minimizing his energy function.
\textcite{HaunertWolff2010AreaAgg} aggregated land-cover areas
based on mixed-integer linear programming.
This method is based on global optimization and minimizes 
type changes as well as a cost for non-compact shapes 
while satisfying constraints on the sizes of the output regions.
\textcite{Haunertwolff2010Building} simplified building
footprints by solving an integer linear program (ILP).
They aimed at minimizing the number of edges in the output under 
the restriction that the simplified buildings must be 
topologically safe,
that is, the selected and extended edges must not intersect with 
each other.
\textcite{Oehrlein2017Aggregation} aggregated the departments 
of France according to the unemployment rates based on integer 
linear programming; they used a cutting-plane method to 
speed up solving their ILP.
\textcite{Funke2017Simplification} simplified 
administrative boundaries based on an ILP.
Their aim was to minimize the number of edges
while keeping the result boundaries close to the original ones
and avoiding any intersection. 
At the same time, every city, which is represented by a point, stays in the same face as before.



%\todo{may cite \textcite{Bader2001Dissertation}}

\begin{figure}[tb]
	\centering
	\includegraphics{AreaAgg_example}
	\caption{The input and a possible output for an instance of 
		our problem}
	\label{fig:AreaAgg_example}
\end{figure}


\mypar{Optimization in \emph{continuous} map generalization}
Optimization becomes more delicate
when we deal with CMG.
In this field, we have requirements 
not only for a specific map, 
but also for relations between maps. 
%
Nevertheless, some optimization techniques 
have been applied to CMG.
In the aforementioned article,
\textcite{Noellenburg2008} used dynamic programming
to match points of two polylines to support morphing
according to some matching cost.
\textcite{sahw-oarps-ICAGW13} used 
mixed-integer linear programming 
to select points of interest.
They required that 
a point, once disappeared, should not show up again 
during zooming out, 
and that any two points should be 
sufficiently far away from each other.
Based on these requirements, 
they wanted to show as many points as possible 
for the given scale interval.
\textcite{Chimani2014Eat} computed a deletion sequence
for a road network by integer linear programming
and efficient approximation algorithms.
They wanted to delete a stroke, 
which is a sequence of edges, at each step
while keeping the remaining network connected.
They assigned each edge a weight, 
and their objective was to maximize the total weight 
over all the road networks, at each step.
\textcite{Peng2013LSA} defined trajectories 
based on LSA for morphing between polylines, 
where LSA is used to mediate between the requirements 
for angles and edge lengths of the intermediate polylines.



\mypar{Problem definition}
The land-cover area is of significant importance on maps.
When users zoom out, some land-cover areas become
tiny polygons, which result in visual clutter.
In order to provide users 
with good visual experience during zooming operations,
we want to remove these tiny polygons.
\textcite{Smith2007MasterMap,Thiemann2018LandCover} 
proposed to split these tiny parcels 
and then to merge the split parts into the neighboring polygons.
\textcite{Meijers2016Split} developed an algorithm 
to split a polygon (the splittee) 
based on constrained Delaunay triangulation.
During splitting, their algorithm is capable of
taking into account
the attractivenesses between the splittee and its neighbors.
When merging, a more attractive neighbor 
will get a larger portion from the splittee.
However, our method does not consider splitting.
We focus on aggregating areas.
A \emph{land-cover map} is a planar subdivision in which each 
area belongs to a
land-cover class or \emph{type}.
Suppose that there are two land-cover maps of 
different scales 
that cover the same spatial region.
We consider the problem of finding a sequence 
of small incremental changes 
that gradually transforms 
the larger-scale map (the \emph{start map}) to 
the smaller-scale map (the \emph{goal map}).
We use this sequence to generate and show land-cover maps at 
intermediate scales (see \fig\ref{fig:AreaAgg_example}).
In this way, we try to avoid large and discrete changes
during zooming.

With the same motivation, a strategy of hierarchical schemes 
has been proposed.
This strategy generalizes a more-detailed representation 
to obtain a less-detailed representation 
based on small incremental changes, 
e.g., the Generalized Area Partitioning tree 
(GAP-tree),
which can be constructed if only the larger-scale map is given
\citep{vanOosterom1995Development} 
or if both the larger-scale map and the smaller-scale map 
are given \citep{HaunertDilo2009}.
Typically, the next change in such a sequence 
is determined locally, 
in a greedy fashion.  
If we insist on finding a sequence that is, 
according to some global measure, optimal, 
the problem becomes complicated.


We assume that there exist many-to-one relationships between the 
areas of the
start map and the areas of the goal map.
This assumption is based on the fact that many algorithms (e.g., \citealp{HaunertWolff2010AreaAgg,vanSmaalen2003,Oehrlein2017Aggregation})
that aggregate land-cover areas
result in many-to-one relationships.
The input and the generalized results of these methods 
can be used as our input.
We term the areas of the goal map \emph{regions}.
That is, every region is the union of a set of areas 
on the start map.
The type of a region may differ from the types of its 
components. 
For example, a small water area together with 
multiple adjacent forest areas may constitute 
a large forest region on the smaller-scale map.
However, we assume that every region, on the goal map, 
contains at least one area of the same type on the start map.
Our assumptions hold if the goal map has been produced 
with an automatic method for area aggregation, 
for example, with the method of \citet{HaunertWolff2010AreaAgg}.
That method produces a land-cover map at a single output scale, 
given a land-cover map at a larger scale.
Although the method of \textcite{HaunertWolff2010AreaAgg}
leads to results of high quality, 
it does not yield land-cover maps at intermediate scales.


To generate a good sequence of maps, 
we must consider the sequence of maps as a whole 
instead of considering each intermediate-scale map independently.
To simplify the problem,
we first consider each region of the goal map 
(with its components on the start map) 
independently of the other regions.
Once we have found an aggregation sequence for each region, 
we integrate all the sequences into an overall sequence 
that transforms the start map into the goal map. 
Our aggregation sequence may be cooperated with
the GAP-face tree \citep{vanOosterom2005},
the map cube model \citep{Timpf1998},
or ScaleMaster \citep{Brewer2007Guidelines,Touya2013ScaleMaster}, 
to support on-the-fly visualization.
Smoothly (dis-)appearing areas can be realized
by integrating our results into the \emph{space-scale cube}
\parencite{vanOosterom2014tGAP,vanOosterom2014tGAPSSC}.

%\todo{may cite \textcite{Suba2014Merge}}

\mypar{Contribution}
We formally define our problem, 
introduce the basic concepts of our method, 
and analyze the size of our model in a worst-case scenario
(see \sect\ref{sec:AreaAgg_Preliminaries}).
We define our cost functions in 
\sect\ref{sec:AreaAgg_CostFunctions}.
Then, we compare three methods for finding aggregation sequences.
First, we develop 
a greedy algorithm (\sect\ref{sec:AreaAgg_Greedy}).
Second, we present a new global optimization approach
based on the \Astar algorithm 
(\sect\ref{sec:AreaAgg_AStar}).
Third, we model our pathfinding problem as
an integer linear program, 
and then solve this program, minimizing our cost function.
Our linear program uses binary (0--1) variables. 
These variables help us to model our problem, 
but in general, it is NP-hard to solve an ILP optimally.
By comparing with the greedy algorithm, 
which is used as a benchmark,
we are able to see whether it is worth 
to use \Astar or the ILP-based algorithm, 
which are more complex and slower.  
Our case study 
uses a data set of the German topographic database ATKIS 
(see \sect\ref{sec:AreaAgg_CaseStudy}).
In the conclusions, we show possible ways to improve our 
methods in \sect\ref{sec:AreaAgg_Conclusions}. 

We do not deal with the simplification of lines in this paper, 
since this can be handled separately from the
aggregation of areas, for example, 
by using one method of 
\textcite{Douglas1973,Saalfeld1999,Wu2004DP}.
Those methods can be used to set up 
the binary line generalized (BLG) tree
\citep{vanOosterom1995Development},
which is a hierarchical data structure that 
defines a gradual line simplification process.



\section{Preliminaries}
\label{sec:AreaAgg_Preliminaries}

We show how to compute an aggregation sequence for one region. 
For a goal map with many regions, 
after having found a sequence for each of them
we ``interleaf'' these aggregation sequences
with respect to the order of the smallest patches 
(see for example \fig\ref{fig:AreaAgg_IntegrateSequence}).
This integration is similar to the merge step in the 
merge sort algorithm; 
see \textcite[pp.~29--37]{Cormen2009}.
To allow us to describe our method more easily,
below we assume that the goal map has only one region.
This region consists of~$n$ land-cover 
areas (components) on the start map. 
In other words, the union of the~$n$ land-cover areas 
is the only region on the goal map.
\begin{figure*}[tb]
	\centering
	\includegraphics[page=1]{AreaAgg_Preliminaries}
	\caption{Integrating two aggregation sequences 
		of different regions: 
		the resulting sequence contains the given sequences 
		as subsequences and 
		always takes the subdivision with smallest patch next.
		The gray arrows show the integration of the two regions.
	}
	\label{fig:AreaAgg_IntegrateSequence}
\end{figure*}


To find a sequence of small changes 
that transforms the start map into the goal map,
we require that 
every change involves only two areas of the current map.
More precisely, in each step the smallest area~$u$ is merged 
with one of its neighbors (adjacent areas)~$v$
such that~$u$ and~$v$ are replaced by their union.
The type of the union is restricted to be 
the type of either~$u$ or~$v$.
If the union uses the type of~$u$, 
we say that area~$v$ is \emph{aggregated into} area~$u$, 
and vice versa. 
How to aggregate exactly is decided by 
optimizing a global cost function.
This requirement ensures that the 
size of the smallest area in the map increases in each step
and thus the sequence reflects a gradual reduction of the 
map's scale.
From another perspective, 
for simplicity we consider 
the smallest area as the least important, 
rather than involving more rules for (non-)importance. 
Even though the requirement reduces the number of possible 
solutions,
there is still enough room for optimization 
since we leave open with
which of its neighbors the smallest area is aggregated.
We term a sequence of changes 
that adheres to our smallest-first requirement simply 
an \emph{aggregation sequence}.


\mypar{Model}
We consider a directed graph $G$, 
which we call the \emph{subdivision graph} 
(see \fig\ref{fig:AreaAgg_SubdivisionName}). 
The node set $V$ of~$G$ contains a node for every
possible map (or \emph{subdivision}), including the start map, all
possible intermediate-scale maps, and the goal map.
The arc set~$E$ of~$G$ contains an arc $(\Pnode, P_{t+1,j})$ 
between any two maps~$\Pnode$ and~$P_{t+1,j}$ in~$V$ 
if~$P_{t+1,j}$ can be reached from~$\Pnode$ 
with a single aggregation operation,
involving the smallest area.
Then, any directed path in~$G$ from the start map to the goal map
defines a possible aggregation sequence.
Our idea is to compute an optimal aggregation sequence through 
computing a minimum-weight path from the start to the goal.
This idea obviously requires that the arc weights are 
set such that a minimum-weight start--goal
path does actually correspond to an aggregation sequence of 
maximum cartographic quality.  
Moreover, putting the idea to practice is far from trivial 
since the graph $G$ can be huge as we will argue below.
We compare a greedy algorithm, \Astar, and an ILP-based algorithm
in finding such paths.
Note that we only know subdivisions 
$\Pstart$ and $\Pgoal$ at the beginning.
We generate a subdivision (node) only when we want to visit it.


\begin{figure}[tb]
	\centering
	\includegraphics[page=2]{AreaAgg_Preliminaries}
	\caption{The subdivision graph~$G$. 
		The nodes of the graph are the subdivisions. 
		There is an arc	from subdivision~\Pnode 
		to subdivision~${P}_{t+1,j}$ 
		if ${P}_{t+1,j}$ is the result of 
		aggregating a smallest area~$p$ of~\Pnode with a 
		neighbor of~$p$.}
	\label{fig:AreaAgg_SubdivisionName}
\end{figure}

\mypar{Notation}
We represent each land-cover area by a polygon with a type.
We denote by $P$ the set of polygons on the start map.
We use~$p$, $q$, $r$, or~$o$ to denote polygons.
A \emph{patch} is a set of polygons whose union is connected. 
The patch also has the property type.
We use~$u$ or~$v$ to denote patches.


Recall that we are dealing with a single region and 
there are~$n$ land-cover areas on the start map in this region. 
Hence, the desired aggregation sequence consists of~$n-1$ steps. 
There are~$n$ subdivisions on a path 
from the start map to the goal map. 
We use~$t \in {T}=\{1,2,\dots,n\}$ to denote \emph{time}. 
When~$t=1$, the subdivision consists of~$n$ patches, 
and there is only one patch remaining when~$t=n$.
The subdivision graph consists of layers~${L}_1,\dots,{L}_n$, 
where layer~${L}_t=\{{P}_{t,1},\dots,{P}_{t,n_t}\}$
contains every possible subdivision~\Pnode\ with~$n-t+1$ patches (see \fig\ref{fig:AreaAgg_ExponentialSize}).

\begin{figure}[tb]
	\centering
	\includegraphics[page=3]{AreaAgg_Preliminaries}
	\caption{An example to show that the size of our problem 
		has an exponential lower bound.}
	\label{fig:AreaAgg_ExponentialSize}
\end{figure}

\mypar{Exponential lower bound}
We now analyze the size of the subdivision graph~$G$.
Consider an instance that consists of 
a start map with $n=2k+1$ rectangles in a row and
a goal map that is simply the union of the $n$ rectangles
(see \fig\ref{fig:AreaAgg_ExponentialSize}).
At layer~$L_{k+1}$, we need to remove~$k$ of the~$2k$ 
intermediate vertical boundaries. 
For this, the number of choices is~$
{{2k}\choose{k}}
= \frac{2k\cdot (2k-1)\cdot \ldots \cdot (k+1)}
{k\cdot (k-1)\cdot \ldots \cdot 1}
\ge 2^k = 2^{(n-1)/2}$.
As a result, the size of our model
has an exponential lower bound.


\section{Cost Functions}
\label{sec:AreaAgg_CostFunctions}

\fig\ref{fig:AreaAgg_SubdivisionName} shows that there are many 
ways to aggregate from the start map to the goal map.
Apparently, some of the ways are more reasonable than 
others.
In our example, we consider 
the sequence~$(P_{1,1}, P_{2,1},P_{3,1},P_{4,1})$ 
more reasonable than 
the sequence~$(P_{1,1}, P_{2,4},P_{3,5},P_{4,1})$.
The reason is that the dark area should not expand so much
as the target color is gray.
We want to provide users with a most reasonable sequence as we 
believe that an unreasonable sequence irritates users.
To find a most reasonable sequence, we introduce cost functions 
in our problem of finding shortest paths.
In the cost functions, we charge a higher penalty 
when an aggregation step is less reasonable.
As a result, by minimizing the overall cost of an aggregation 
sequence, we find a most reasonable sequence.

It is difficult to define exactly what reasonable means 
because users may have different preferences.
Four preferences are discussed by
\textcite{Cheng2006}; see \fig\ref{fig:AreaAgg_Preferences}.
They are aggregating a small land-cover area into one 
that isolates the area 
(\fig\ref{fig:AreaAgg_Preferences}b), 
that is the largest neighbor 
(\fig\ref{fig:AreaAgg_Preferences}c), 
that shares the longest boundary 
(\fig\ref{fig:AreaAgg_Preferences}d), or 
that has the most similar type
(\fig\ref{fig:AreaAgg_Preferences}e).
To keep our aggregation problem independent of user preferences,
our cost function takes two aspects into account: 
one is based on semantics and 
the other is based on shape.
%
In terms of semantics, we require that 
the \emph{type} of a land-cover area changes 
as little as possible.
This requirement means that we prefer, for example, aggregating 
an area with type \emph{swamp} into an area with type \emph{wet 
	ground} rather than into an area with type \emph{city}.
%
In terms of shape, we hope to have areas 
which are as \emph{compact} as possible.
Our argument is that an area is easier 
to be identified by human being 
if it is more compact.
%
We also consider the total \emph{length} 
of the interior boundaries as an alternative compactness:
we consider a subdivision~\Pnode as more compact than
another subdivision~$P_{t,i'}$ if the total length of the interior boundaries of~\Pnode is less than that of~$P_{t,i'}$.
We add this alternative because we want to make a comparison 
involving an ILP, 
where a \emph{linear} cost function must be used.
Note that most compactness measures
are \emph{not} linear;
see, for example, \textcite{Maceachren1985,Li2013Compactness}.
Although the length of interior boundaries 
is not sufficiently precise 
to describe compactness \parencite{Young1988},
it is often used as a reasonable replacement
when compactness is considered in an ILP
(e.g., \citealp{Minas2016,Wright1983}).
\textcite{HaunertWolff2010AreaAgg} employed 
the centroids of a set of land-cover areas.
One of their costs is the sum of the distances 
from the centroids to a \emph{reference point}.
The reference point is the centroid
that minimizes the sum.
The sum of the distances can also be computed linearly.
We use the length of interior boundaries 
instead of the distance because the former 
is more relevant to the shapes of the patches.
Also, \textcite{Harrie2015Readability} showed that
longer lines generally yield lower map readability.

\begin{figure}[tb]
	\centering
	\includegraphics[page=1]{AreaAgg_CostFunctions}
	\caption{Aggregating land-cover areas according to different 
		preferences.
		Aggregating a small land-cover area into one 
		that isolates the area (b), 
		that is the largest neighbor (c), 
		that shares the longest boundary (d), or 
		that has the most similar type (e).}
	\label{fig:AreaAgg_Preferences}	
\end{figure}


\subsection{Cost of type change}
\label{sec:AreaAgg_f_type}

We define the cost of type change as follows.
Suppose that we are at the step of aggregating 
from subdivision~$P_{s,i}$ to subdivision~$P_{s+1,j}$. 
In this step, patch~$u$ is aggregated into patch~$v$
(see \figs\ref{fig:AreaAgg_FirstStep}a 
and \ref{fig:AreaAgg_FirstStep}b).
We denote the types of the two patches by~$T(u)$ and~$T(v)$. 
We define the cost of type change of this step by
\begin{equation}
\label{eq:f_type}
f_\mathrm{type}(P_{s,i},P_{s+1,j})=\frac{A_{u}}{A_R}
\cdot
\frac{d(T(u),T(v))}{T_{\max}},
\end{equation}
where~$A_u$ is the area of patch~$u$, 
and~$A_R$ is the area of region~$R$.
We use~$A_R$ and~$T_{\max}$ 
to normalize the cost of type change. 
Constant~$T_{\max}$, the maximum cost over all type changes,  
is known from the input. 
The input specifies, for each pair~$(T_1,T_2)$ of types, 
cost~$d(T_1,T_2)$ of changing type~$T_1$ to type~$T_2$.
Specifically, we denote by~$T_\mathrm{goal}$ the type of 
the patch on the goal map.
For simplicity, we use a metric 
as the cost function of type change
(see \sect\ref{sec:AreaAgg_CaseStudy}); 
see \textcite[chapter 2]{Choudhary1992Elements}
for the definition of a metric.
A metric distance is symmetric, 
which is different from the unsymmetric one used in
\textcite{Dilo2009tGAP}.
In their definition, for example, 
the distance from type \emph{building} to type \emph{road} is~$0.5$,
but the distance is~$0$ the other way around.

For path $\Pistar=(P_{1,i_1}, P_{2,i_2}, \dots, 
P_{\tstar,i_{\tstar}})$,
we define the cost of type change over the steps so far by
\begin{equation}
\label{eq:g_type}
g_\mathrm{type}(\Pistar)=
\sum_{s=1}^{t-1}f_\mathrm{type}(P_{s,i_s},P_{s+1,i_{s+1}})
\end{equation}

\begin{figure}[tb]
	\centering
	\includegraphics[page=2]{AreaAgg_CostFunctions}
	\caption{An aggregation step, 
		where patch~$u$, the dark patch at the top,
		is aggregated into patch~$v$.
		Figures~(c) and~(d) respectively show the number of 
		edges and the lengths of the interior polylines after 
		the aggregation.}
	\label{fig:AreaAgg_FirstStep}	
\end{figure}

\subsection{Cost of compactness}
\label{sec:AreaAgg_f_comp}

We use the compactness definition of \citet{Frolov1975}, 
i.e., the compactness value for patch~$u$ is
\begin{equation*}
\label{eq:comp}
c(u)=\frac{2 \sqrt{\pi A_u}}{l_u},
\end{equation*}
where~$A_u$ and~$l_u$ are 
the area and the perimeter. 
For a subdivision \Psnode, we denote by~$C(\Psnode)$ 
the set of the patches' compactnesses.

We wish to maximize the sum of the average compactnesses 
over all intermediate maps,
while our objective will be
minimizing a cost function.
To adapt the average compactness to our methods, 
we define and minimize a cost related to compactness.
Recalling that at time~$s$ there are~$n-s+1$ patches,
we define the cost of compactness for subdivision~\Psnode as
\begin{equation}
\label{eq:f_comp}
f_{\mathrm{comp}}(\Psnode)=
\frac{1-\frac{1}{n-s+1} \sum_{c\in C(\Psnode)}c}{n-2},
\end{equation}
where we use values~$n-s+1$ and~$n-2$ to normalize 
the cost of compactness.

For path $\Pistar$ (see \sect\ref{sec:AreaAgg_f_type}),  
we define the cost of compactness over all 
intermediate maps
(that is, neglecting~$P_{1,1}$ 
and the last subdivision in the path) by
\begin{equation}
\label{eq:g_comp}
g_\mathrm{comp}(\Pistar)=
\sum_{s=2}^{\tstar-1}f_{\mathrm{comp}}(P_{s,i_s}).
\end{equation}


\subsection{Cost of length}
\label{sec:AreaAgg_costlength}

We denote the set of interior boundaries 
for a subdivision~$\Psnode$ by~$B(\Psnode)$.
The cost in terms of interior length of 
this subdivision is defined as
\begin{equation}
\label{eq:f_length}
f_{\mathrm{lgth}}(\Psnode)=
\frac{\big(\sum_{b\in B (\Psnode)} 
	|b|\big)\big/D(s)}{n-2}, 
\end{equation} 
where 
\begin{equation}
\label{eq:AreaAgg_Norm}
D(s)=\frac{n-s}{n-1} \sum_{b\in B (\Pstart)} |b|.
\end{equation}
Function~$D(s)$ computes the ``expected'' total length of 
the interior boundaries at time~$s$,
where we expect that this total length decreases linearly
according to time~$s$.
In special,~$D(1) = \sum_{b\in B (\Pstart)} |b|$
and~$D(n) = 0$.
Similarly to \eq\ref{eq:f_comp}, 
we use~$D(s)$ and~$n-2$ to normalize 
the cost of length.


For path~$\Pistar$ (see \sect\ref{sec:AreaAgg_f_type}), 
we define the cost of length over all 
intermediate maps 
(that is, neglecting~$P_{1,1}$ 
and the last subdivison in the path) by
\begin{equation}
\label{eq:g_length}
g_\mathrm{lgth}(\Pistar)=
\sum_{s=2}^{\tstar-1}f_{\mathrm{lgth}}(P_{s,i_s}).
\end{equation}



\subsection{Combining cost functions}
\label{sec:AreaAgg_Combining}
In order to minimize the type change 
and maximize the compactness,
we combine the two cost functions as
\begin{equation}
\label{eq:g_1}
g_1(\Pistar)= (1-\lambda)g_\mathrm{type}(\Pistar)
+\lambda g_\mathrm{comp}(\Pistar),
\end{equation}
where~$\lambda \in [0,1]$ is a parameter 
to assign importances 
of~$f_\mathrm{type}$ and~$f_\mathrm{comp}$.
We simply use~$\lambda=0.5$ in our experiments. 
We want to find a path~$\Pi$ from~$\Pstart$ to~$\Pnode$ 
that minimizes, among all such paths,~$g(\Pistar)$.
Slightly abusing notation, we denote the cost of
this optimal path from~${P}_{\mathrm{start}}$ to~\Pnode 
by~$g_1(\Pnode)$.
Using $g_1(\Pnode)$, we compare a greedy algorithm and \Astar
in finding optimal sequences for area aggregation.

As said before, we want to make a comparison involving
integer linear programming.
Therefore, we also introduce a combination
that can be computed linearly, as
\begin{equation}
\label{eq:g_2}
g_2(\Pistar)= (1-\lambda)g_\mathrm{type}(\Pistar)
+\lambda g_\mathrm{lgth}(\Pistar).
\end{equation}
We compare the greedy algorithm, \Astar, and an ILP-based 
algorithm using $g_2$.


\section{A Greedy Algorithm}
\label{sec:AreaAgg_Greedy}
At any time~$t$,
our greedy algorithm aggregates the smallest patch
with one of its neighbors.
Suppose that patch~$u$ is the smallest patch,
and patch~$v$ is one of~$u$'s neighbors.
We aggregate~$u$ into~$v$ 
if the type distances fulfill that
$d(T(u), T_\mathrm{goal}) \ge d(T(v), T_\mathrm{goal})$;
otherwise, we aggregate~$v$ into~$u$.
Recall that when we aggregate a patch into another patch,
the union takes the type of the latter.
We compute the costs of aggregating the smallest patch 
with each of the neighbors, 
and select the aggregation that costs least.
In other words, the smallest patch is aggregated with 
its most \emph{compatible} neighbor.
This idea is the same as that of \textcite{vanOosterom2005},
except we have a different definition for the compatible.
In accordance 
with our two combinatorial costs in 
\sect\ref{sec:AreaAgg_Combining},
we define two cost functions.
Suppose that we are at the step of 
aggregating from subdivision $P_{s,i}$ to subdivision 
$P_{s+1,j}$.
The first cost function is 
\begin{equation}
\label{eq:f_1}
f_1(P_{s,i},P_{s+1,j})=
(1-\lambda)f_\mathrm{type}(P_{s,i},P_{s+1,j})
+\lambda f_{\mathrm{comp}}(P_{s+1,j}). \nonumber
\end{equation}
The second cost function is
\begin{equation}
\label{eq:f_2}
f_2(P_{s,i},P_{s+1,j})=
(1-\lambda)f_\mathrm{type}(P_{s,i},P_{s+1,j})
+\lambda f_{\mathrm{lgth}}(P_{s+1,j}). \nonumber
\end{equation}



\section{Using the \Astar Algorithm}
\label{sec:AreaAgg_AStar}


%\mypar[inline]{Motivate \Astar by comparison with Dijkstra.}


We try to find a shortest path,
from the start map to the goal map,
using \Astar \citep{Hart1968}.  
As the graph~$G$---our search space---can be of exponential size 
(see \sect\ref{sec:AreaAgg_Preliminaries}), 
we avoid computing the whole graph explicitly.
To save time and memory, 
we generate a subdivision \Pnode only 
when we are going to visit it.
\Astar uses a clever best-first search 
to find a shortest path 
from subdivisions~$\Pstart$ to~$\Pgoal$.  
For subdivision \Pnode,
\Astar considers the exact cost of a shortest path 
from~\Pstart to~\Pnode 
and estimates the cost to get from~\Pnode to~\Pgoal.  
\Astar can be seen as a refinement of Dijkstra's algorithm
where nodes are explored earlier 
if they are estimated to be closer to the goal.

More formally, we define $g(\Pnode)$ to be 
the exact cost of a shortest path from~\Pstart to~\Pnode, 
and we define $h(\Pnode)$ to be the estimated cost 
to get from \Pnode to~\Pgoal. 
Then the (estimated) total cost at node~\Pnode is
\begin{equation}
\label{eq:CostTotal}
F(\Pnode)=g(\Pnode)+h(\Pnode).
\end{equation}
We use either~$g_1$ (\eq\ref{eq:g_1}) 
or~$g_2$ (\eq\ref{eq:g_2})
for~$g(\Pnode)$.
If~$h(\Pnode)$ is always bounded from above 
by the exact cost of a shortest path from~\Pnode to~\Pgoal, 
\Astar is guaranteed to find a shortest path from~\Pstart 
to~\Pgoal, 
that is, an optimal aggregation sequence.  
Using the estimation~$F$ (\eq\ref{eq:CostTotal}), 
\Astar is able to reduce the search space.
The better the estimation part~$h$, 
the more \Astar can reduce the search space.
In the following, we show how we compute
estimated cost $h(\Pnode)$.

To narrow down the search space, 
we set up estimation functions 
for type change (\sect\ref{sec:AreaAgg_h_type}), 
compactness (\sect\ref{sec:AreaAgg_h_comp}), 
and length (\sect\ref{sec:AreaAgg_h_length}). 
These functions are meant to direct \Astar towards the goal.
Since the number of subdivisions can be exponential, 
it can happen that we run out of main memory 
before we find an optimal solution. 
To handle this problem, we overestimate the cost. 
Overestimation is a popular technique while using \Astar,
in order to find a feasible solution. 
For example, \textcite{Pohl1973} overestimated 
using dynamic weighting. 
We propose another strategy that fits our problem. 
We first try finding an optimal solution by \Astar. 
If we fail to find one after 
we have visited a predefined number~$W$ of nodes of graph~$G$,
then we restart.
In the retrying, we overestimate the first~$K$ steps 
starting at each node
(see \sects\ref{sec:AreaAgg_h_type}, \ref{sec:AreaAgg_h_comp},
and~\ref{sec:AreaAgg_h_length}).
We may need to increase~$K$ and retry several times 
until we find a feasible solution.
We define~$K$ by
\[
K= 2^k -1,
\]
where~$k\ge 0$ is the number of retryings.
As~$K\le n-1$, it holds that~$k \le \log_2 n +1$, 
which means  that we will 
retry~$\lceil \log_2 n\rceil$ times at most.
When we are at time $t$, there are at most $n-t$ steps 
to arrive at the goal map.
We define the number of practical overestimation steps as
\[
K'= \min \{K, n-t\}.
\]
Whenever overestimating ($k\geq1$), 
\Astar cannot guarantee optimality anymore.


\subsection{Estimating the cost of type change}
\label{sec:AreaAgg_h_type}

To find a lower bound of the cost of type change, 
we simply assume that 
every patch will be aggregated into a patch with type~\Tgoal.
As long as the cost of type change is a metric, this aggregation 
strategy indeed yields a lower bound.
%
For subdivision~\Pnode, let
$(\Pnode=P_{t,i_t}, P_{t+1,i_{\tstar+1}} \dots,P_{n,i_n}=\Pgoal)$,
be the path that always changes the type of a smallest patch 
to~\Tgoal.
Then the estimated cost of type change is
\begin{equation}
\label{eq:h_type}
h_\mathrm{type}(\Pnode)=
\sum_{s=t}^{n-1}f_\mathrm{type}(P_{s,i_s},P_{s+1,i_{s+1}}).
\end{equation}

As an example, for \fig\ref{fig:AreaAgg_FirstStep}b, we compute 
$h_\mathrm{type}$ 
according to the aggregation sequence of 
\fig\ref{fig:AreaAgg_h_type}.
Note that the step from $P_{2,i_2}$ to 
$P_{3,i_3}$ 
in 
\fig\ref{fig:AreaAgg_h_type} is impossible in reality 
because the dark patch cannot be aggregated into patch~$v$
as they are not neighbors. 
However, this aggregation is allowed for estimation 
because we may find a shortest path as long as 
the estimated cost is no more than 
the exact cost of a shortest path.
In order to overestimate, we multiply the estimated cost of the 
first $K'$ steps by $K$.
\fo\ref{eq:h_type} is revised to
\begin{equation}
\label{eq:o_type}
h_\mathrm{type}(\Pnode)=
K\sum_{s=t}^{t+K'-1}f_\mathrm{type}(P_{s,i_s},P_{s+1,i_{s+1}})+
\sum_{s=t+K'}^{n-1}f_\mathrm{type}(P_{s,i_s},P_{s+1,i_{s+1}}).
\nonumber
\end{equation}


\begin{figure*}[tb]
	\centering
	\includegraphics[page=1]{AreaAgg_Estimation}
	\caption{An ``aggregation sequence'' for computing 
		$h_{\mathrm{type}}$, 
		based on the aggregation result of 
		\fig\ref{fig:AreaAgg_FirstStep}b}
	\label{fig:AreaAgg_h_type}
\end{figure*}


\subsection{Estimating the cost of compactness}
\label{sec:AreaAgg_h_comp}

Our way to estimate the cost of compactness 
is based on regular polygons.
The more edges a regular polygon has, the more compact it is.
We assume that at each step 
we aggregate the two patches that are the least compact.
Moreover, we assume that the shared boundary of the two patches  has the least number of edges.
We use $\mathcal{N}_\mathrm{ext}$ to 
denote the edge number of the region's exterior boundaries.
As the exterior boundaries will not be changed by aggregation, 
$\mathcal{N}_\mathrm{ext}$ 
is a constant. 
%
Note that the boundary between two patches 
is not necessarily connected 
(see for example the dark boundary with three edges 
in \fig\ref{fig:AreaAgg_h_comp}a).
We denote by~$B(\Psnode)$ 
the set of interior boundaries at time~$s$,
and denote by~$b_\mathrm{min}(\Psnode)$ 
the boundary with the smallest number of edges.
For our estimation, the set of interior boundaries 
at time~$s+1$ 
is~$B(P_{s+1,j})= B(\Psnode)-{\{b_\mathrm{min}(\Psnode)\}}$. 
The estimated number of all the edges for 
such a subdivision~$P_{s+1,j}$ is
\begin{equation}
\label{eq:LeftEdgeNum}
\mathcal{N}_{s+1,j}=
\mathcal{N}_\mathrm{ext} + \sum_{b \in B(P_{s+1,j})} \|b\|.
\end{equation}
where $\|b\|$ is the number of edges of boundary~$b$.

\begin{figure*}[tb]
	\centering
	\includegraphics[page=2]{AreaAgg_Estimation}
	\captionof{figure}{An "aggregation" sequence for 
		computing 
		$h_\mathrm{comp}$ (see 
		\eq\ref{eq:o_comp}) based on the 
		number of edges. 
		At each step we remove the boundary with the fewest 
		edges. 		
		The numbers represent the 
		numbers of the interior boundaries' edges.
		This example corresponds to the aggregation step in 
		\fig\ref{fig:AreaAgg_FirstStep}.
	}
	\label{fig:AreaAgg_h_comp}
\end{figure*}

From subdivision~\Psnode to subdivision~$P_{s+1,j}$,
we get a new patch because of aggregation.
The new patch is certainly less compact than a regular polygon 
with~$\mathcal{N}_{s+1,j}$ edges.
In order to estimate the compactness of the new patch, 
we assume that 
the new patch has the shape of a regular polygon 
with~$\mathcal{N}_{s+1,j}$ edges 
(see \eq\ref{eq:LeftEdgeNum}).
A regular polygon with~$\mathcal{N}$ edges has compactness
\begin{equation*}
\label{eq:comp_regular}
c_\mathrm{reg}(\mathcal{N})=
\sqrt{\frac{\pi}{\mathcal{N}} \bigg/
	\tan{\frac{\pi}{\mathcal{N}}}}.
\end{equation*}
Note that~$c_\mathrm{reg}(\mathcal{N})$ increases 
with increasing~$\mathcal{N}$.
A patch with~$\mathcal{N}_{s+1,j}$ edges has
compactness~$c_\mathrm{reg} (\mathcal{N}_{s+1,j})$.
According to our previous assumption, 
at each step we are able to aggregate the two patches 
that are the least compact in the subdivision.
We denote the compactness values of the two patches
by~$c_\mathrm{min}(\Psnode)$ and~$c'_\mathrm{min}(\Psnode)$.
Recall that we use~$C(\Psnode)$ to denote 
the set of compactness values of the patches
in subdivision~$\Psnode$ 
(see \sect\ref{sec:AreaAgg_f_comp}).
Then the set of compactness values 
for subdivision~$P_{s+1,j}$ is 
\begin{equation}
\label{eq:compsetnew}
C(P_{s+1,j})=
C(\Psnode)\cup 
\{c_\mathrm{reg} (\mathcal{N}_{s+1,j})\}
\setminus \{c_\mathrm{min}(\Psnode), c'_\mathrm{min}(\Psnode)\}.
\nonumber
\end{equation}
We compute the estimated average compactness 
by calculating the average 
of the values in set~$C(P_{s+1,j})$.
Finally, we compute the estimated cost of compactness for subdivision~$P_{s+1,j}$ by \eq\ref{eq:f_comp}.

For subdivision~\Pnode, let
$(\Pnode=P_{t,i'_t}, P_{t+1,i'_{\tstar+1}}, 
\dots,P_{n,i'_n}=\Pgoal)$
be the path that always removes the two smallest compactnesses
and gains a compactness of the constructed regular polygon.
The estimated cost of compactness is
\begin{equation}
\label{eq:h_comp}
h_\mathrm{comp}(\Pnode)=
\sum_{s=t}^{n-1}f_\mathrm{comp}(P_{s,i'_s}). \nonumber
\end{equation}
When overestimating, we assume that 
each patch in the subdivision is extremely noncompact,
that is, each patch has compactness~$0$.
One may ask if this assumption is too much.
It is indeed too much for one subdivision, 
but it is just fine for the whole sequence
as we overestimate for only a certain number of subdivisions.
Based on this assumption,
the cost of compactness is $f_\mathrm{comp}(P_{s,i_s})=1/(n-2)$,
according to \eq\ref{eq:f_comp}.
If we overestimate $K'$ steps, 
we revise the estimated cost of compactness to
\begin{equation}
\label{eq:o_comp}
h_\mathrm{comp}(\Pnode)=
\sum_{s=t}^{t+K'-1}\frac{1}{n-2}+
\sum_{s=t+K'}^{n-1}f_\mathrm{comp}(P_{s,i'_s}).
\end{equation}



\subsection{Estimating the cost of length}
\label{sec:AreaAgg_h_length}

At time $s$, there are $n-s+1$ patches.
There can be as few as $n-s$ interior boundaries.
In order to find a lower bound of the cost of length,
we keep only the necessary number, $n-s$, 
of shortest boundaries at each step
(see \fig\ref{fig:AreaAgg_h_length}).
Then we compute the estimated cost of length according to 
\eq\ref{eq:f_length}.

For subdivision~\Pnode, let
$(\Pnode=P_{t,i''_t}, P_{t+1,i''_{\tstar+1}}, \dots,P_{n,i''_n}=\Pgoal)$
be the path that always keeps 
the necessary number of shortest interior boundaries.
The estimated cost of length is
\begin{equation}
\label{eq:h_length}
h_\mathrm{lgth}(\Pnode)=
\sum_{s=t}^{n-1}f_\mathrm{lgth}(P_{s,i''_s}).
\end{equation}
When overestimating,
we use the interior length of the current map as the cost of 
length for $K'$ steps even though we are removing 
interior boundaries.
As a result, we revise \fo\ref{eq:h_length} to
\begin{equation}
\label{eq:o_length}
h_\mathrm{lgth}(\Pnode)=
\sum_{s=t}^{t+K'-1}f_\mathrm{lgth}(P_{t,i})+
\sum_{s=t+K'}^{n-1}f_\mathrm{lgth}(P_{s,i''_s}). \nonumber
\end{equation}


\begin{figure*}[htb]
	\centering
	\includegraphics[page=3]{AreaAgg_Estimation}
	\caption{An "aggregation" sequence for computing 
		$h_\mathrm{lgth}$ 
		(see Equation~\ref{eq:h_length}) 
		based on the lengths of boundaries. 
		At each step we keep the necessary number 
		of boundaries with least lengths in order to find a 
		lower bound of edge length of the interior boundaries 
		$l_\mathrm{int}(t)$. 
		The numbers represent the 
		lengths of the interior boundaries.
		This example corresponds to the aggregation step in 
		\fig\ref{fig:AreaAgg_FirstStep}
	}
	\label{fig:AreaAgg_h_length}
\end{figure*}


\subsection{Combining estimated costs}
\label{sec:AreaAgg_CombinationEstimated}
In accordance 
with our two combinatorial costs in 
\sect\ref{sec:AreaAgg_Combining},
we define two estimated-cost functions:
\begin{equation}
\label{eq:h_1}
h_1(\Pnode)=
(1-\lambda)h_\mathrm{type}(\Pnode)
+\lambda h_{\mathrm{comp}}(\Pnode), \nonumber
\end{equation}
and
\begin{equation}
\label{eq:h_2}
h_2(\Pnode)=
(1-\lambda)h_\mathrm{type}(\Pnode)
+\lambda h_{\mathrm{lgth}}(\Pnode). \nonumber
\end{equation}

\section{Integer Linear Programming}
\label{sec:AreaAgg_ILP}

\emph{Linear programming} is a method 
to optimize a \emph{linear objective}
subject to a set of \emph{linear constraints},
with some \emph{variables}.
For example, suppose that we are selling coffee.
We have~$3.5\,$kg of coffee powder and~$10\,$kg of water.
We mix the powder and the water to provide coffee with 
two kinds of intensity in terms of mass:~$40\%$ and~$20\%$.
The profits of the two kinds of coffee 
are respectively~$5\,$\euro and~$4\,$\euro.
We want to maximize the total profit of selling coffee.
If we offer respectively~$x\,$kg and~$y\,$kg 
of the two kinds of coffee,
then~$x$ and~$y$ are our variables.
Our objective is to
$$
\mathrm{maximize} 	\quad	 5x+4y
$$
To provide~$x\,$kg of coffee with intensity~$40\%$,
we need to use~$0.4x\,$kg of coffee powder 
and~$0.6x\,$kg water.
Analogously, it consumes~$0.2y\,$kg of coffee powder 
and~$0.8y\,$kg of water to produce~$y\,$kg of coffee
with intensity~$20\%$.
Therefore, we have four constraints:
\begin{align*}
0.4x+0.2y	&\le 3.5,				\\
0.6x+0.8y 	&\le 10,	\text{~and}			\\
x,y			&\ge 0.
\end{align*}
With the objective and the constraints, 
we have set up a \emph{linear program} (LP).
We observed that 
all the feasible solutions, i.e., pairs of~$(x,y)$,
fall in the gray area 
shown in \fig\ref{fig:AreaAgg_ILPIllustration}a.
Drawing a line with slope~$-\frac{5}{4}$,
we see that every pair of~$(x,y)$ lying on the line
yields the same result for~$5x+4y$, 
the profit we want to maximize.
For example, every pair of~$(x,y)$ lying on the dashed line
in \fig\ref{fig:AreaAgg_ILPIllustration}a 
yields profit~$40\,$\euro.
If we move the dashed line to the upper right,
then we are able to achieve a larger value for~$5x+4y$. 
In order to maximize the profit, 
we move the dashed line to the upper right as much as possible
and at the same time make sure that 
the it still intersects with the gray area.
Note that if the dashed line 
does not intersect with the gray area,
then there is no feasible pair of~$(x,y)$ 
on the dashed line anymore.
As a result, we get the optimal solution 
when the dashed line hits point~$A$,
where the profit is~$5 \cdot 4 + 4 \cdot 9.5 =58\,$\euro.
\textcite{Karmarkar1984LP}
proved that an LP can be solved in a polynomial time.

Now we change our problem a bit.
We assume that we want to sell coffee in bottles,
where each bottle contains exactly~$1\,$kg of coffee 
with intensity either~$40\%$ or~$20\%$.
Our question becomes how many bottles of each kind of coffee
we should sell in order to maximize the profit.
If we sell the two kinds of coffee 
respectively~$x'$ and~$y'$ bottles,
then our problem becomes:
\begin{alignat*}{3}
&&\text{maximize} 	\quad	&& 5x'+4y' 		&			\\
&&\text{subject to} \quad	&& 0.4x'+0.2y'	&\le 3.5, 	\\
&&					\quad	&& 0.6x'+0.8y' 	&\le 10, 	\\
&&					\quad	&& x', y' 		&\ge 0, 	\\
&&\text{and} 		\quad	&& x', y'		&\in \mathbb{Z}.
\end{alignat*}
For this problem, only the pairs of~$(x,y)$ 
represented by the gray dots in \fig\ref{fig:AreaAgg_ILPIllustration}b 
are feasible solutions
(point~$A$ is no longer an optimal solution in this case).
In order to maximize our profit,
we should move the dashed line to the upper right 
as much as possible
and at the same time make sure that 
it hits at least one of the gray dots.
To solve such a problem is known as
\emph{integer linear programming},
which is NP-complete.
Despite the fact, there are
mathematical solvers yielding optimal solutions
for some NP-complete problems in reasonable time
\parencite{Haunert2017Label}.
Also, by using these solvers we profit 
from every improvement, by their producers,
for the same class of problems
\parencite{Haunert2017Label}.
The general form of an \emph{integer linear program} (ILP) is
\begin{alignat*}{3}
&&\text{maximize} 	\quad&& \bm{C}^\mathrm{T}\bm{X}	&		\\
&&\text{subject to} \quad&& \bm{EX}			&\le \bm{H}, 	\\
&&					\quad&& \bm{X} 			&\ge \bm{0}, 	\\
&&\text{and}		\quad&& \bm{X} 			&\in \mathbb{Z}^I.
\end{alignat*}
where vector~$\bm{X}$ represents integer variables, 
vector~$\bm{C} \in \mathbb{R}^I$, 
vector~$\bm{H} \in \mathbb{R}^J$,
and~$\bm{E}$ is a $(J \times I)$-matrix over the reals.
Furthermore, if we require that
$$
\bm{X} 	\in \{0,1\}^I,
$$
then we have only binary variables for an ILP.
Binary variables are important 
because they occur regularly in optimizations
\parencite[section~9.2]{bradley1977applied}.
Also, an ILP with general (bounded) integer variables 
can always be translated to an ILP with binary variables
\parencite[section~2.3]{Williams2009Integer}.
To find optimal sequences for area aggregation, 
we are going to use binary variables in our ILP.

\begin{figure*}[tb]
	\centering
	\includegraphics[page=4]{AreaAgg_ILP}
	\caption{Example of linear programming (a)
		and integer linear programming (b). 
	}
	\label{fig:AreaAgg_ILPIllustration}
\end{figure*}

We want to compare the \Astar algorithm with 
integer linear programming in finding 
optimal solutions for our aggregation problem. 
Since integer linear programming
can handle only linear constraints, 
we define the compactness of a subdivision as 
the length of the subdivision's interior boundaries.
That is, we use cost function~$g_2$ (see \eq\ref{eq:g_2}).
Our basic idea is to formalize the problem of 
finding a shortest path as an ILP.
Then we solve this ILP by minimizing the total cost.
We define the \emph{center} of a patch as the polygon 
to which other polygons in the same patch are assigned. 
At the beginning, every patch consists of only one polygon, 
and this polygon is the center of the patch.
When we aggregate patch~$u$ into patch~$v$, 
all the polygons of~$u$ are assigned to the center of~$v$,
and the type of~$u$'s polygons 
are changed to the type of $v$'s center.
In the following, we show how to formalize our problem 
as an ILP.
For simplicity, we sometimes denote by \emph{patch~$r$} the 
patch using polygon~$r$ as the center at time~$t$.


\subsection{Variables}
\label{sub:AreaAgg_variables}

Our problem is to decide centers for polygons to be assigned.
Each question of type 
``Is polygon~$p$ assigned to center~$r$?''
can be answered with ``yes'' or ``no''.
Hence, we use binary ($0$--$1$) variables.
We need five sets of variables
in order to formulate our pathfinding problem as an ILP.
%
The first set of variables is used to tell the program 
our rules for area aggregation.
\begin{figure*}[tb]
	\centering
	\includegraphics[page=1]{AreaAgg_ILP}
	\caption{Some examples of variables~$x$,~$y$, and~$z$ for our ILP. 
		The arrows in the figures show the aggregation steps and
		the dotted lines represent the removed boundaries by the aggregation steps.
	}
	\label{fig:AreaAgg_Variables}
\end{figure*}
We introduce the variable 
\begin{flalign*}
&\myquad[4]
\embrd[V]{x_{t,p,r}} \in
\embld[U]{\{0,1\}} \qquad 
\forall t\in T, \forall p,r \in P&
\end{flalign*}
with the intended meaning~$x_{t,p,r}=1$ if and only if 
polygon $p$ is assigned to polygon $r$ at time $t$. 
If a polygon is a center at time~$t$, 
then the polygon must be assigned to itself, 
that is, $x_{t,r,r}=1$
(see \fig\ref{fig:AreaAgg_Variables} for some examples).

Our second set of variables is used to compute the
cost of type change.  We introduce
\begin{flalign*}
&\myquad[4]
\embrd[V]{y_{t,p,o,r}} \in
\embld[U]{\{0,1\}} \qquad 
\forall t\in T\setminus \{1\}, \forall p,o,r \in P&
\end{flalign*}
with the intended meaning~$y_{t,p,o,r}=1$ if and only if 
polygon~$p$ is assigned to center $o\in P$ at time~$t-1$ 
and to center~$r$ at time~$t$.
Specifically, $y_{t,p,o,o}=1$ means that
polygon~$p$ is assigned to the same center 
at times~$t-1$ and~$t$ 
(see \fig\ref{fig:AreaAgg_Variables}).

We need a third set of variables 
for computing the cost of length.
We introduce
\begin{flalign*}
&\myquad[4]
\embrd[V]{z_{t,p,q,r}} \in
\embld[U]{\{0,1\}} \qquad 
\forall t\in T\setminus \{1,n\}, \forall p,q,r \in P&
\end{flalign*}
with the intended meaning~$z_{t,p,q,r}=1$ 
if and only if at time~$t$
polygons~$p$ and~$q$ are both in patch~$r$.
In this case, their common boundary should be removed
(see \fig\ref{fig:AreaAgg_Variables}).
When variable~$z_{t,p,q,r}=1$ and~$p=q$,
we define the length of their common boundary to be~$0$ 
because we shall not remove any.
Note that time $t\in T\setminus \{1,n\}$.
We do not need~$z_{t,p,q,r}$ for time~$t=1$ 
because at the time
there are no two polygons in the same patch.
Namely, it always holds that~$z_{1,p,q,r}=0$, 
which does not help in our ILP.
We do not need~$z_{t,p,q,r}$ for time~$t=n$
because in this case
all the polygons are in the same patch.
The situation that~$z_{n,p,q,r}=1$ always holds
does not help in our ILP, either.

We use a fourth set of variables
to guarantee contiguity of each patch. 
In other words, we aggregate two patches 
only when they are neighbors (adjacent).
We introduce
\begin{flalign*}
&\myquad[4]
\embrd[V]{c_{t,p,o,r}} \in
\embld[U]{\{0,1\}} \qquad 
\forall t\in T\setminus \{n-1,n\}, 
\forall p,o,r \in P \text{~with}~o\ne r,&
\end{flalign*}
with the intended meaning~$c_{t,p,o,r}=1$ 
if and only if at time~$t$
polygon~$p$ is assigned to center~$o$ 
and~$p$ has a neighbor assigned to center~$r$
(see for example \fig\ref{fig:AreaAgg_Variables_Neighbor}).
We do not need a variable~$c_{t,p,o,r}$ for time~$t=n-1$ 
because then there are only two patches left 
and they must be neighbors.

Our last set of variables is needed to 
enforce that
every aggregation step involves a smallest patch. 
We define
\begin{flalign*}
&\myquad[4]
\embrd[V]{w_{t,o}} \in
\embld[U]{\{0,1\}} \qquad 
\forall t\in T\setminus\{n\}, \forall o \in P&
\end{flalign*}
with $w_{t,o}=1$ meaning 
that patch~$o$ is a smallest patch at time~$t$.

In total, the number of variables
in our ILP formulation is~$O(n^4)$.

\begin{figure}[tb]
	\centering
	\includegraphics[page=2]{AreaAgg_ILP}
	\caption{There are two patches, 
		which respectively use polygons~$o$ and~$r$ 
		as their centers.
		Polygons in the same patch 
		are separated by dotted lines.
		Polygon~$p$, in patch~$o$, 
		has two neighbors assigned to center~$r$,
		i.e., polygons~$q_1$ and~$q_2$.
		In this case, patches~$o$ and~$r$ are neighbors 
		and can be aggregated.
	}
	\label{fig:AreaAgg_Variables_Neighbor}
\end{figure} 


\subsection{Objectives}
\label{sub:AreaAgg_objectives}

Our aim is to minimize a weighted sum of the two costs, 
the cost of type change and the cost of length,
that is, we
\begin{equation}
\label{eq:ilpcost}
\mathrm{minimize} \quad 
(1-\lambda)F_\mathrm{type} +\lambda F_\mathrm{lgth},
\nonumber
\end{equation}
where~$\lambda$, as in \eq\ref{eq:g_1}, 
is a parameter 
to assign importances 
of~$f_\mathrm{type}$ and~$f_\mathrm{comp}$.
According to the objective we introduced in
\sect\ref{sec:AreaAgg_f_type},
the total cost of type change is defined as
\begin{flalign*}
%\label{eq:F_type}
&\myquad[4]
\embrd[V]{F_\mathrm{type}} =
\sum_{t=2}^{n} \sum_{p\in P} \sum_{o\in P} \sum_{r\in P}
\left(\frac{a_p}{A_R} \cdot
\frac{d(T(o),T(r))}{T_{\max}}\cdot 
y_{t,p,o,r}\right), & 
\end{flalign*}
where, similar to \eq\ref{eq:f_type}, 
$a_p$ is the area of polygon~$p$,
$A_R$ is the area of the region, 
and~$T(o)$ and~$T(r)$ are the types of centers~$o$ and~$r$.


We also try to minimize the overall interior lengths of all the 
intermediate subdivisions.
As discussed in \sect\ref{sec:AreaAgg_costlength},
we need a cost that can be computed linearly.
We use the length of the interior boundaries 
as an alternative to compactness.
Recall that $B(\Pstart)$ is the set of interior boundaries 
at time $t=1$ (see \sect\ref{sec:AreaAgg_costlength}).
The sum of the normalized length of 
the remaining interior boundaries at all times 
can be computed by
\begin{flalign}
\label{eq:F_length_raw}
&\myquad[4]
\embrd[V]{F_\mathrm{lgth}} =
\frac{1}{n-2} \sum_{t=2}^{n-1} 
\frac{\sum_{b\in B (\Pstart)} |b| - 
	\frac{1}{2} \sum_{p \in P}\sum_{q \in P}\sum_{r \in P} 
	\big(|b_{pq}|\cdot z_{t,p,q,r}\big)} {D(t)}, &
\end{flalign}
where variable~$b_{pq}$ represents the common boundary 
between polygons~$p$ and~$q$. 
We define the lengths of the common boundary to be~$0$
(i.e., $|b_{pq}|=0$) if $p=q$ 
because no boundary is removed in this case.
Function~$D(t)$, defined by \eq\ref{eq:AreaAgg_Norm}, 
is used to normalize the cost of length.
As in \eq\ref{eq:f_length}, 
denominator~$n-2$ is used to balance 
between the costs of type change and length.
Integrating~$D(t)$ into \eq\ref{eq:F_length_raw}, we have
\begin{flalign*}
%\label{eq:F_length}
&\myquad[4]
\embrd[V]{F_\mathrm{lgth}} =
\frac{n-1}{n-2} \sum_{t=2}^{n-1}
\left(
\frac{1}{n-t} -
\frac{\sum_{p \in P}\sum_{q \in P}\sum_{r \in P} 
	\big(|b_{pq}|\cdot z_{t,p,q,r}\big)}
{2(n-t)\sum_{b\in B (\Pstart)} |b| }
\right). &
\end{flalign*}




\subsection{Constraints}
\label{sub:AreaAgg_Constraints}

In order to formulate our aggregation problem as an ILP, 
we restrict the variables introduced in \sect\ref{sub:AreaAgg_variables} 
by setting up constraints.
Recall that the intended meaning 
of~$x_{t,p,r}=1$ is that, at time~$t$, 
polygon~$p$ is assigned to center~$r$. 
Our first constraint is that, at time~$t$, 
polygon~$p$ is assigned to 
exactly one center. 
To this end, we require that
\begin{flalign}
\label{eq:CstrOneCenter}
&\eqquad
\embrd{\sum_{r\in P} x_{t,p,r}} =\embld{1} 
\inquad \forall t \in {T}, \forall p \in P. &
\end{flalign}


The next constraint is that polygon~$r$ is available to be 
assigned by other polygons only when~$r$ is a center.
In our case, if polygon~$r$ is a center, 
then it must be assigned to itself,
that is, $x_{t,r,r}=1$.
If~$r$ is not a center, then $x_{t,r,r}=0$.
In either case,
\begin{flalign}
\label{eq:CstrAssign}
&\eqquad
\embrd{x_{t,p,r}} \leq \embld{x_{t,r,r}}
\inquad \forall t \in {T}, \forall  p, r \in P. &
\end{flalign}


Aggregating one patch into another results in 
the number of centers decreasing by~$1$.
We make sure that exactly one patch 
is aggregated into another by 
specifying the number of centers
for each point in time, that is,
\begin{flalign}
%\label{eq:CstrCountCenter}
&\eqquad
\embrd{\sum_{r\in P} x_{t,r,r}} =
\embld{n-t+1} \inquad
\forall t \in {T}, &
\end{flalign}
where polygon $r$ is a center at time~$t$ if and only if 
$x_{t,r,r}=1$.

A center can disappear, but will never reappear afterwards.
Hence,
\begin{flalign}
\label{eq:CstrNoReappear}
&\eqquad
\embrd{x_{t,r,r}} \le 
\embld{x_{t-1,r,r}} \inquad 
\forall t \in {T}\setminus \{1\},
\forall r \in P.&
\end{flalign}


On the start map, 
there are some polygons with  goal type $T_\mathrm{goal}$
(see definition in \sect\ref{sec:AreaAgg_h_type}).
At time $t=n$, all polygons are aggregated into one patch.
This patch must have the goal type, $T_\mathrm{goal}$.
In other words, the center of this patch must be one of the 
polygons with the 
goal type on the start map:
\begin{equation}
\label{eq:CstrType}
\sum_{r\in P\colon T(r)=T_\mathrm{goal}}
x_{n,r,r}=1,
\end{equation}
where $T(r)$ is the type of polygon $r$ at time $t=1$.

Next, we use the binary variables~$y_{t,p,o,r}$
introduced in \sect\ref{sub:AreaAgg_variables}.  
Recall that the intended meaning of~$y_{t,p,o,r}=1$ is that 
polygon~$p$ is assigned to center~$o$ at time~$t-1$ 
and to center~$r$ at time~$t$.
To enforce this, we use two types of constraints.

First, if polygon~$p$ is assigned 
to center~$o$ at time~$t-1$ ($x_{t-1,p,o}=1$)
and assigned to center~$r$ at time $t$
($x_{t,p,r}=1$), then $y_{t,p,o,r}=1$.  This is expressed by
\begin{flalign}
\label{eq:CstrY1}
&\eqquad
\embrd{y_{t,p,o,r}} \geq 
\embld{x_{t-1,p,o}+x_{t,p,r}-1} \inquad
\forall t \in {T} \setminus \{1\}, 
\forall p, o, r \in P.&
\end{flalign}

Second, if~$p$ is not assigned to~$o$ at~$t-1$ ($x_{t-1,p,o}=0$)
or~$p$ is not assigned to~$r$ at time $t$ ($x_{t,p,o}=0$),
then $y_{t,p,o,r}=0$.
This is expressed by
\begin{flalign}
\label{eq:CstrY2}
&\eqquad
\begin{array}{@{}l}
\embrd{y_{t,p,o,r}} \le \\
\embrd{y_{t,p,o,r}} \le 
\end{array}
\embld{\hspace{0.5pt}
	\begin{array}{@{}l}
	x_{t-1,p,o} \\
	x_{t,p,r}
	\end{array}
	\bigg\} 
}
\inquad \hspace{0.5pt}
\forall t\in T\setminus \{1\}, 
\forall	p,o,r \in P.&	
\end{flalign}



%Third, if centers $o$ and $r$ are identical, 
%then the center of polygon $p$ is not changed.
%In this case, $y_{t,p,o,r}=0$, which is presented by
%\begin{equation}
%\label{eq:CstrY3}
%y_{t,p,o,r}=0 \qquad
%\forall t \in {T} / \{n\}, 
%\forall p, o, r \in P.
%\end{equation}

In \sect\ref{sub:AreaAgg_variables},
we introduced the binary variables~$z_{t,p,q,r}$.
Recall that the intended meaning of $z_{t,p,q,r}=1$ is that, at
time~$t$, polygons~$p$ and~$q$ are both in patch~$r$.
To enforce this, we need three types of constraints.

First, if two polygons~$p$ and~$q$ are assigned 
to center~$r$ at time~$t$ ($x_{t,p,r}=x_{t,q,r}=1$),
then~$z_{t,p,q,r}=1$.  This is expressed by
\begin{flalign}
\label{eq:CstrZ1}
&\eqquad
\embrd{z_{t,p,q,r}} \geq 
\embld{x_{t,p,r}+x_{t,q,r}-1} \inquad
\forall t \in T \setminus \{1,n\}, 
\forall p, q, r \in P.&
\end{flalign}

Second, if not both~$p$ and~$q$ are assigned to~$r$,
then $z_{t,p,q,r}=0$.  This is expressed by
\begin{flalign}
\label{eq:CstrZ2}
&\eqquad
\begin{array}{@{}l}
\embrd{z_{t,p,q,r}} \le  \\
\embrd{z_{t,p,q,r}} \le 
\end{array} 
\embld{\hspace{0.5pt}
	\begin{array}{@{}l}
	x_{t,p,r} \\
	x_{t,q,r}
	\end{array}
	\bigg\} 
}
\inquad \hspace{0.5pt}
\forall t \in T \setminus \{1,n\}, 
\forall p, q, r \in P.&	
\end{flalign}
Note the correspondence between 
constraints~(\ref{eq:CstrY1}) and~(\ref{eq:CstrZ1})
and between 
constraints~(\ref{eq:CstrY2}) and~(\ref{eq:CstrZ2}).

Third, we introduce an abbreviation that will be helpful to 
express
the last type of constraint involving variables~$z_{t,p,q,r}$:
\begin{flalign}
\label{eq:CstrZabbrv}
&\eqquad
\embrd{z_{t,p,q}} = 
\embld{\sum_{r \in P}z_{t,p,q,r}}
\inquad 
\forall t\in T \setminus \{1,n\}, 
\forall	p,q \in P,&
\end{flalign}
where the reason we do not need~$z_{t,p,q}$ for~$t=1$ or~$t=n$
is the same as for $z_{t,p,q,r}$
(see \sect\ref{sub:AreaAgg_variables}).
Note that $z_{t,p,q}$ expresses whether, at time~$t$, 
polygons~$p$ and~$q$ are 
in the same patch ($z_{t,p,q}=1$) or not ($z_{t,p,q}=0$).
%
Note further that 
constraints~(\ref{eq:CstrOneCenter}) and~(\ref{eq:CstrZ2}) 
ensure that~$p$ and~$q$ can only be assigned 
to one common patch, that is, $z_{t,p,q} \le 1$.

Now we use our new variables~$z_{t,p,q}$
to express the following requirement. 
If two polygons have been aggregated into one patch, 
they will always be in the same patch 
at later times~-- although the center of their common patch 
can change after the point in time 
when they were aggregated first. 
In other words, as a function of time~$t$, 
$z_{t,p,q}$ is monotonically increasing:
\begin{flalign}
\label{eq:CstrTogether}
&\eqquad
\embrd{z_{t,p,q}} \ge 
\embld{z_{t-1,p,q}} \inquad
\forall t \in \{3,4,\ldots,n-1\},  
\forall p, q \in P.&
\end{flalign}

The problem of ensuring contiguity inside a patch 
when using integer linear programming 
has received considerable attention.
Usually, a subdivision is represented by a graph
(see \fig\ref{fig:AreaAgg_Variables_Graph}).
%
\textcite{Zoltners1983Territory} regarded each node as a center.
For each center, they found a shortest path 
to each of the other nodes.
Then they required that a center can be assigned 
by a node only if 
at least one immediate predecessor of the node 
in the shortest path had been assigned to the center.
Although this requirement makes their problem easier to be 
solved,
it excludes many feasible patches.
%
\textcite{Williams2002Contiguous} found an optimal spanning tree 
with a user-specified number of nodes, say,~$i$.
In order to ensure contiguity, this method requires that 
there must be~$i-1$ edges in the spanning tree.
%
For a given center, \textcite{Cova2000_Contiguity} 
were able to find all the contiguous patches.
In their method, when a node is to be assigned to a center, 
a path from the node to the center was demanded 
that each node is assigned to the center.
%
Similarly, \textcite{Shirabe2005Contiguity} modeled 
the contiguity problem as a network flow.
He required that there must be a path so that 
some fluid could flow from a node to a sink (center).
%
\textcite{Oehrlein2017Aggregation} utilized a method based 
on \emph{vertex separators}.	
Given center~$r$ and node~$p$, 
a separator is a set of nodes
such that any path from~$r$ to~$p$ 
will contain at least one node of the set.
The contiguity between center~$r$ and node~$p$ is ensured 
if each of the separators contains 
at least one node assigned to the center.
%
The last four ideas can be adapted into our method
as we do not wish to exclude any possible solutions.
However, we use an idea
that is more intuitive for our problem
since we aggregate step by step.


\begin{figure}[tb]
	\centering
	\includegraphics[page=3]{AreaAgg_ILP}
	\caption{The graph of a subdivision.
		Each polygon of the subdivision is represented as a node 
		in the graph.
		There is an edge between two nodes
		if the corresponding two polygons are adjacent.
	}
	\label{fig:AreaAgg_Variables_Graph}
\end{figure} 

We aggregate two patches only if they are neighbors.
To ensure this, we need the binary variables~$c_{t,p,o,r}$
introduced in \sect\ref{sub:AreaAgg_variables}.
Recall that the intended meaning of~$c_{t,p,o,r}=1$ 
is that, at time~$t$, 
polygon~$p$ of patch~$o$ has 
at least one neighboring polygon in patch~$r$.
To enforce this behavior of~$c_{t,p,o,r}$, 
we need four types of constraints.

First, at time~$t$, polygon~$p$ must actually be assigned 
to center~$o$,
\begin{flalign}
\label{eq:CstrC_Part}
&\eqquad
\embrd[C]{c_{t,p,o,r}} \le 
\embld[D]{x_{t,p,o}} \inquad
\forall t 	 \in T\setminus \{n-1,n\},  
\forall p, o, r \in P \text{~with}~o\ne r.&
\end{flalign}

Second, at time~$t$, polygon~$p$ has at least 
one neighbor that is assigned to center~$r$,
\begin{flalign}
\label{eq:CstrC_Neighbor}
&\eqquad
\embrd[C]{c_{t,p,o,r}} \le 
\embld[D]{\sum_{q\in N_\mathrm{nbr}(p)} x_{t,q,r}} \inquad
\forall t 	 \in T\setminus \{n-1,n\},  
\forall p, o, r \in P \text{~with}~o\ne r,&
\end{flalign}
where~$N_\mathrm{nbr}(p)$ represents 
the set of polygons adjacent to~$p$.


Third, if polygon~$p$ is in patch~$o$ and~$p$ has at least 
one neighbor in patch~$r$, then we must enforce~$c_{t,p,o,r}=1$,
\begin{flalign}
\label{eq:CstrC_Positive}
&\eqquad
\begin{array}{@{}l}
\embrd[C]{c_{t,p,o,r}} \ge  \\
\embrd[C]{} %an empty place
\end{array} 
\embld[D]{\hspace{0.5pt}
	\begin{array}{@{}l}
	x_{t,p,o} + x_{t,q,r} -1 \\
	~ %an empty place
	\end{array}
}
\inquad \hspace{0.5pt}
\begin{array}{@{}l}
\forall t 	 \in T\setminus \{n-1,n\},\\
\forall p, o, r \in P \text{~with}~o\ne r,
\forall q \in N_\mathrm{nbr}(p).
\end{array} &	
\end{flalign}


Fourth, if we aggregate patch~$o$ into patch~$r$
from time~$t-1$ to time~$t$, 
then $y_{t,o,o,r}=1$.
In this case, we must make sure that 
the two patches are actually neighbors,
\begin{flalign}
\label{eq:CstrC_Agg}
&\eqquad
\embrd[C]{y_{t,o,o,r}} \le 
\embld[D]{\sum_{p\in P} c_{t-1,p,o,r}} \inquad
\forall t 	 \in T\setminus \{1,n\},  
\forall o, r \in P \text{~with}~o\ne r.&
\end{flalign}


If we do not require that 
each aggregation step must involve a smallest patch,
then we only need constraints
(\ref{eq:CstrOneCenter})--(\ref{eq:CstrC_Agg})
explained so far 
and variables~$x_{t,p,r}$, $y_{t,p,o,r}$, $z_{t,p,q,r}$,
and~$c_{t,p,o,r}$.
If we insist on involving a smallest patch at each step,
then we need more variables and more constraints.


\mypar{Aggregation involving a smallest patch}
In order to make sure that 
each of our aggregation steps involves a smallest patch,
we need another type of variable,~$w_{t,o}$.
Recall that the intended meaning of~$w_{t,o}=1$ is that 
polygon~$o$ is the center of 
a smallest patch at time~$t$.
We will use this to enforce that
this patch is involved in the aggregation step 
from time~$t$ to~$t+1$.
At any time~$t$, 
we pick exactly one smallest patch and aggregate it with 
one of its neighbors;
we do not care whether or not the neighbor is a smallest one.
Therefore, we have
\begin{flalign}
\label{eq:CstrSOneSmallest}
&\eqquad
\embrd[S]{\sum_{o\in P}w_{t,o}} =
\embld[T]{1} \inquad 
\forall t \in {T}\setminus \{n\}.&
\end{flalign}

Assume that patch~$o$ is the smallest patch
which is involved in the aggregation step from~$t$ to~$t+1$,
and that we are aggregating patches~$o$ and~$r$.
There can be two cases.
Patch~$o$ is aggregated into patch~$r$, 
or the other way around.
In the first case, we have variable~$y_{t+1,o,o,r}=1$,
and in the second case, we have~$y_{t+1,r,r,o}=1$.
Either of the two constraints implies 
that polygon~$o$ is indeed a center at time~$t$, 
that is,~$x_{t,o,o} =1$.
In order to enforce that
the aggregation step involves patch~$o$,
we must make sure that~$y_{t+1,o,o,r}=1$ or~$y_{t+1,r,r,o}=1$ 
when $w_{t,o}=1$.
Consequently, we use the constraint
\begin{flalign}
\label{eq:CstrSInvolveSmallest}
%\displaystyle doesn't work for \setminus
%if we do \embld{\sum_{r\in P \setminus \{o\}}
&\eqquad
\embrd[S]{w_{t,o}} \le 
\embld[T]{\displaystyle\sum_{r\in P \setminus \{o\}} \left(y_{t+1,o,o,r} +y_{t+1,r,r,o}\right)} \inquad
\forall t \in {T}\setminus \{n\}, 
\forall o \in P. &
\end{flalign}

Now we need to make sure that
patch~$o$ with $w_{t,o}=1$ is indeed 
a smallest patch at time~$t$.
We define variable~$A_{t,r}$ as the area of 
patch~$r$ at time~$t$, 
that is,
\begin{flalign*}
&\eqquad
\embrd[S]{A_{t,r}} = 
\embld[T]{\sum_{p\in P} a_p \cdot x_{t,p,r},} &
\end{flalign*}
where~$a_p$ is the area of polygon~$p$ 
and is a constant (as viewed by the ILP).
Area~$A_{t,r}$ is positive 
if and only if polygon~$r$ is a center at time~$t$,
that is, if and only if~$x_{t,r,r}=1$.
We define a very large number~$M$ 
to help us construct the corresponding constraints.
It suffices to set~$M$ to 
the area of the whole region, 
i.e., $M=A_R$ (see \eq\ref{eq:f_type}). 
We require that
%\begin{flalign}
%\label{eq:CstrSIndeedSmallest}
%&\eqquad
%\embrd[S]{A_{t,o} - M(1-w_{t,o})} \le
%\embld[T]{A_{t,r}+M(1-x_{t,r,r})} \inquad
%\forall t 	 \in {T}\setminus \{n\}, 
%\forall o, r \in P \text{~with}~o\ne r. &
%\end{flalign}
%\begin{flalign}
%\label{eq:CstrSIndeedSmallest}
%&\eqquad
%\embrd[S]{A_{t,o}} -
%\embld[T]{M(1-w_{t,o}) \le A_{t,r}+M(1-x_{t,r,r})} \inquad
%\forall t 	 \in {T}\setminus \{n\}, 
%\forall o, r \in P \text{~with}~o\ne r. &
%\end{flalign}
\begin{equation}
A_{t,o} - M(1-w_{t,o}) \le A_{t,r}+M(1-x_{t,r,r}) \inquad
\forall t 	 \in {T}\setminus \{n\}, 
\forall o, r \in P \text{~with}~o\ne r.
\end{equation}
This constraint takes effect only when~$w_{t,o}=1$ 
and~$x_{t,r,r}=1$, which indicates that 
patch~$o$ is smaller than or equal to 
all the other existing patches at time~$t$.

In order to compute an aggregation sequence 
involving a smallest patch at each step, 
we need all the five types of variables and 
constraints~(\ref{eq:CstrOneCenter})--(\ref{eq:CstrSIndeedSmallest}).
In total, the number of constraints is~$O(n^4)$.

\section{Case Study}
\label{sec:AreaAgg_CaseStudy}
We have implemented our methods 
based on C\# (Microsoft Visual Studio~$2015$) 
and ArcObjects SDK 10.4.1.
We used the IBM ILOG
CPLEX Optimization Studio~$12.6.3.0$ to solve our ILP.
Our prototype is open access on 
GitHub\footnote{
	\url{https://github.com/IGNF/ContinuousGeneralisation},
	Accessed: 2018-Jun-18.}.
Our computer ran
Windows~$7$ on a~$3.3\,$GHz dual core CPU with~$8\,$GB RAM.
We measured processing time 
by the built-in C\# class \emph{Stopwatch}.
As required by ArcObjects SDK~$10.4.1$,
we specified our program to run on the $32$-bit platform. 
We added a post-build task about ``largeaddressaware''
in Microsoft Visual Studio so that 
we were able to use up to $3\,$GB of main memory\footnote{
	The details of the setting can be found at 	
	\url{http://stackoverflow.com/questions/2597790/can-i-set-largeaddressaware-from-within-visual-studio},
	Accessed: 2017-Nov-01.}.
Our CPLEX version may declare an optimal solution
while it is not really optimal.
To fix this problem, we had to disable 
both primal and dual presolve reductions\footnote{
	For more details about the problem, see
	\url{http://www-01.ibm.com/support/docview.wss?uid=swg1RS02094},
	Accessed: 2017-Nov-12.}.

We tested our method on a data set 
from the German topographic database ATKIS DLM~$50$. 
The data set represents the place 
``Buchholz in der Nordheide'' at scale~$1:50{,}000$. 
Our start map is the result of collapsing areas 
by \citet[pp.~61--66]{haunert2008f}.
The start map has~$5{,}537$ polygons. 
Our goal map was generalized from the start map 
by~\citet{HaunertWolff2010AreaAgg}, setting the scale 
at~$1:250{,}000$ (see \fig\ref{fig:AreaAgg_Data}). 
The goal map has~$734$ polygons, 
which means that there are~$N=734$ regions.
The distribution of region sizes is shown in \fig\ref{fig:AreaAgg_NumRegion}.
%
We used a tree-based method introduced by 
\citet{Rada1989SemanticMetric} 
to define the distances of the types;
the distance is the ``number of edges'' that
we need to travel from one node to another 
node in the tree of type hierarchy\footnote{
	More information about land-cover types can be found at 
	\url{http://www.atkis.de/dstinfo/dstinfo2.dst_gliederung},
	Accessed: 2017-Nov-01.}
(see \fig\ref{fig:AreaAgg_TypeDistances}). 
For example, the distance from type \emph{village}
to type \emph{fence} is~$2$, 
to type \emph{street} is~$4$, and
to type \emph{farm land} is~$6$.
In this tree, the maximum distance is~$6$, 
which means~$T_\mathrm{max}=6$ for \eq\ref{eq:f_type}.
According to \citet{Rada1989SemanticMetric}, 
the resulting cost function for type change is a metric.


\begin{figure}[tb]
	\captionsetup[subfigure]{labelformat=empty}
	\begin{subfigure}[b]{.49\textwidth}
		\centering
		\includegraphics[page=1]{AreaAgg_Data}
		\caption{Start map, $5{,}537$ polygons, \\
			at scale $1:50{,}000$}
	\end{subfigure}
	\hfill
	\begin{subfigure}[b]{.49\textwidth}
		\centering
		\includegraphics[page=2]{AreaAgg_Data}
		\caption{Goal map, $734$ polygons, \\
			at scale $1:250{,}000$}
	\end{subfigure}
	%
	%	\par\vspace{\intextsep} %Leave a gap between the two figures
	\par\vspace{\baselineskip} %Leave a gap	figures
	%
	\begin{subfigure}{\textwidth}
		\centering
		\includegraphics[page=3]{AreaAgg_Data}
		\caption{The 20 land-cover types appearing in our data}
	\end{subfigure}
	\caption{The data of our case study.}
	\label{fig:AreaAgg_Data}
\end{figure}

\begin{figure}[tb]
	\centering
	\includegraphics[page=4]{AreaAgg_Data}
	\caption{Distribution of region sizes:
		the $y$-axis shows how many regions of a given size range are contained in our data set.}
	\label{fig:AreaAgg_NumRegion}
\end{figure}

\begin{figure}[tb]
	\centering
	\includegraphics{AreaAgg_TypeDistances}
	\caption{The tree of type hierarchy used in our case study.
		For example, the distance between 
		types \emph{village} and \emph{farm land} is~$6$.}
	\label{fig:AreaAgg_TypeDistances}
\end{figure}

\subsection{Using costs of type change and compactness}
\label{sec:AreaAgg_CaseStudy1}

As illustrated in \sect\ref{sec:AreaAgg_Combining},
we compare \Astar and the greedy algorithm using $g_1(\Pnode)$,
which is a combination of the costs 
of type change and compactness (see \eq\ref{eq:g_1}).
For \Astar, we overestimated 
whenever we could not find a solution after 
having visited~$W$ nodes 
(see \sect\ref{sec:AreaAgg_AStar}).
We tried $W=200{,}000$ and $W=400{,}000$
(if we could use more main memory, 
then we could test by using some larger~$W$). 
The results are shown in 
Table~\ref{tab:AreaAgg_CaseStudy1_Statistics}.
%
Comparing to \Astar, 
the greedy algorithm visited 
fewer nodes and arcs in the graph $G$ 
and used much less time.
However, 
\Astar managed to find solutions with lower total cost,
$117.3$ (or~$117.2$), 
which is~$2.8\%$ less than 
the total cost of the greedy algorithm,~$120.7$.
%
When~$W=200{,}000$,
we are sure that 
we have found optimal solutions 
for~$702$ of the~$734$ regions ($94.3\%$),
while the greedy algorithm solved only~$407$ of the~$702$ regions to optimality.
For the other~$32$ regions, we have found feasible solutions
by both algorithms
(see column \#over in 
Table~\ref{tab:AreaAgg_CaseStudy1_Statistics}).
Although some of the feasible solutions may also be optimal,
we cannot verify only from the cost values.
\begin{table*}[tb]
	%\small	
	\centering
	\caption{A comparison of the greedy algorithm and \Astar		
		when using cost function~$g_1$ (see \eq\ref{eq:g_1}).
		For \Astar, we used two settings, 
		i.e.,~$W=200{,}000$ and~$W=400{,}000$.
		%
		Symbol~\#over is the number and percentage of regions
		(out of $N=734$) that needed overestimation.
		%
		Variable~$k_\mathrm{sum}$ is 
		the total number of repetitions.
		%
		Symbols~\#nodes and~\#arcs are the total 
		numbers of nodes and arcs that \Astar visited.  
		(For instances where we needed overestimation, 
		only the final attempt was counted.)
		%
		$\sum g_\mathrm{type}$, $\sum g_\mathrm{comp}$, 
		and~$\sum g_1$
		respectively denotes the sums of~$g_\mathrm{type}(\Pgoal)$,
		$g_\mathrm{comp}(\Pgoal)$, and~$g_1(\Pgoal)$ 
		over all the~$734$ instances 
		(see \eqs\ref{eq:g_type}, \ref{eq:g_comp}, 
		and~\ref{eq:g_1}).
		%
		The percentage in the
		time column is the fraction of the total
		runtime spent on solving the instances
		that needed overestimation.
	}
	\label{tab:AreaAgg_CaseStudy1_Statistics}
	\setlength{\tabcolsep}{0.7ex}
	\begin{tabular}{lCRCCCCCd{3.7}}
		\toprule
		\multicolumn{1}{l}{Methods} &
		\multicolumn{1}{c}{\#over} &
		\multicolumn{1}{r}{$k_\mathrm{sum}$} &  
		\multicolumn{1}{c}{\#nodes} & 
		\multicolumn{1}{c}{\#arcs} & 
		\multicolumn{1}{c}{$\sum g_\mathrm{type}$} & 
		\multicolumn{1}{c}{$\sum g_\mathrm{comp}$} & 
		\multicolumn{1}{c}{$\sum g_1$} & 
		\multicolumn{1}{c}{Time (min)} \\ 
		\midrule
		Greedy 	&     &     & 5.5\cdot 10^3 & 4.8\cdot 10^3 
		& 53.2 & 188.2 & 120.7 & 2'0\\
		%
		\Astar$_{\!\!200{,}000}$	&  32~(4.4\%) & 102 &  3.6\cdot 
		10^6 
		&    5.7\cdot 10^6	
		& 51.4 & 183.2 & 117.3 & 75'8~(90.5\%)\\
		%
		\Astar$_{\!\!400{,}000}$	&  29~(4.0\%) &   89 &  6.5\cdot 
		10^6 
		&    9.8\cdot 10^6
		& 51.4 & 183.1 & 117.2 & 137'7~(93.9\%) 
		\\ \bottomrule			
	\end{tabular}
\end{table*}

In accordance with \sect\ref{sec:AreaAgg_AStar}, 
for region with ID~$i$ 
we define~$k_i$ as the least number of repetitions 
that we do to find a feasible solution. 
We define the total number of repetitions 
as~$k_\mathrm{sum}=\sum_{i=1}^N k_i$; 
recall that~$N$ is the number of regions.
After increasing~$W$ to~$400{,}000$, 
\Astar found optimal aggregation sequences 
for only three more regions, 
but $k_\mathrm{sum}$ decreased quite a bit, 
from~$102$ to~$89$. 
The numbers of regions that needed certain
overestimation steps are shown in 
\fig\ref{fig:AreaAgg_OverStats}. 
Besides, \Astar visited more arcs and nodes, 
used more time, 
but got (slightly) less cost when increasing~$W$ to~$400{,}000$.
Although the number of regions 
that needed overestimation is relatively small, 
\Astar spent most of the running time 
on those few regions:~$4.4\%$ and~$4.0\%$ of the regions 
cause~$90.5\%$ and~$93.9\%$ of the total running time, 
respectively (see Table~\ref{tab:AreaAgg_CaseStudy1_Statistics}).

\begin{figure}[tb]
	\centering
	\includegraphics[page=2]{AreaAgg_CaseStudy1_Plot}
	\caption{The numbers of regions where \Astar was 
		forced to use the given overestimation parameters
		in order to find a solution 
		without exploring more than 
		$W \in \{200{,}000;~400{,}000\}$ 
		nodes of the subdivision graph.}
	\label{fig:AreaAgg_OverStats}
\end{figure}

The details of some regions are presented in 
Table~\ref{tab:CostsInDetail}.
According to the entries with overestimation factor $K_i=0$, 
we often have 
ratios~$R_\mathrm{type}=1$ and~$R_\mathrm{comp}>1$.
When factor~$K_i=0$, we did not overestimate for region~$i$.
The estimated cost must be smaller or equal to the exact cost,
which results in~$R_\mathrm{type}\ge 1$ 
and~$R_\mathrm{comp}\ge 1$.
Ratio~$R_\mathrm{type} = 1$ means that our estimation for the 
cost of type change is the best.
A larger~$R_\mathrm{comp}$ means that our estimation for the 
cost of shape is poorer.

\begin{table*}[tb]
	\caption{The costs in detail of some regions, 
		where~$W=200{,}000$.  
		Parameters~$n$ and~$m$ are the numbers of patches and 
		adjacencies on the start map, respectively.
		Parameter $K$ is the overestimation factor, 
		defined in \sect\ref{sec:AreaAgg_Preliminaries}. 
		We evaluate the quality 
		of our estimations for type change and 
		compactness by listing the numbers~
		$R_\mathrm{type}=g_\mathrm{type}(\Pgoal)
		/h_\mathrm{type}(\Pstart)$ and~
		$R_\mathrm{comp}=g_\mathrm{comp}(\Pgoal)
		/h_\mathrm{comp}(\Pstart)$. 
		Note that if~$h_\mathrm{type}(\Pstart)=0$, 
		we define~$R_\mathrm{type}=1$.
		The marked entries are discussed in the text.
	}
	\label{tab:CostsInDetail}
	\centering
	\includegraphics[page=3]{AreaAgg_CaseStudy1_Plot}
\end{table*}

According to columns~$n$ and~$K$ of \tab\ref{tab:CostsInDetail},
\Astar managed to find optimal solutions 
for all the regions with fewer than~$15$ polygons,
and only found feasible solutions 
for any region with more than~$21$ polygons.
Among the~$702$ regions that \Astar solved to optimality,
the greedy algorithm failed to 
find optimal solutions for~$295$ regions.
Solutions of the greedy algorithm cost
at most~$41.7\%$ more than solutions of \Astar;
for region~$85$, the greedy algorithm yields 
a solution of cost~$0.777$, 
while the solution of \Astar has cost~$0.548$
(see \fig\ref{fig:AreaAgg_CaseStudy1_Rg85}).
As the patches in the two sequences are the same,
the two results have the same cost of compactness.
The main difference is the choice of the first step, 
from~$8$ patches to~$7$.
When aggregating the smallest patch on the start map
with the surrounding patch,
our greedy algorithm chooses the type 
which is closer to the goal type.
In this case, the smallest patch has type~$5112$, and the 
surrounding one has~$2112$.
The type of the goal map is~$4102$.
According to \fig\ref{fig:AreaAgg_TypeDistances},
type distance~$d(5112,4102)=4$ and~$d(2112,4102)=6$.
As a result, our greedy algorithm uses~$5112$ 
as the type for the new patch. 
This choice is a big mistake 
because the type of the largest patch on the start map
will have to be changed twice during the aggregation.
These assignments cost more than the sequence obtained by 
\Astar, where the largest patch on the start map is changed to 
the target type directly.

Among the $32$~regions that 
\Astar failed to solve optimally,
the greedy algorithm outperformed \Astar 
for~$15$ regions ($46.9\%$).
%
Among these, solutions of the greedy algorithm 
cost at most~$15.9\%$ less than solutions of \Astar;
for region~$543$, the greedy algorithm yields 
a solution of cost~$0.112$, 
while the solution of \Astar costs~$0.134$. 
For this instance, \Astar used overestimation parameter~$K=7$
(marked in \tab\ref{tab:CostsInDetail}).
\fig\ref{fig:AreaAgg_CaseStudy1_Rg543} shows 
some intermediate results obtained by
\Astar and the greedy algorithm.
Interestingly, the two methods produced the same sequence until 
there were~$8$ patches left.
Then due to the overestimation, \Astar did some bad moves
because the bad aggregation sequence still seemed better 
than other sequences.
In contrast, the greedy algorithm was looking for locally good 
aggregations.
%
Among the~$32$ regions that \Astar failed to solve optimally,  solutions of the greedy algorithm cost at most~$17.4\%$ 
more than solutions of \Astar; 
for region~$155$, the greedy algorithm yields 
a solution of cost~$0.372$, while
the solution of \Astar costs~$0.317$
(marked in \tab\ref{tab:CostsInDetail}).

Finally, an optimal aggregation sequence of region~$53$
(third-last row in \tab\ref{tab:CostsInDetail})
obtained by \Astar
is shown in \fig\ref{fig:AreaAgg_CaseStudy1_Rg53}.

\begin{figure}[tb]
	\centering
	\includegraphics[page=1]{AreaAgg_CaseStudy1}
	\caption{Aggregation sequences of region~$85$ 
		obtained by \Astar and the greedy algorithm.
		In order to save space, we did not show the results when 
		there are~$4$ or~$5$ patches, 
		which one can easily deduce.
		The numbers indicate the numbers of patches.}
	\label{fig:AreaAgg_CaseStudy1_Rg85}
\end{figure}

\begin{figure}[tb]
	\centering
	\includegraphics[page=2]{AreaAgg_CaseStudy1}
	\caption{Some intermediate subdivisions of region~$543$ 
		obtained by \Astar and the greedy algorithm.
		In the sequence obtained by \Astar, 
		a pair of circles or a pair of squares indicates that
		the two parts are actually in the same patch.
		The numbers indicate the numbers of patches.
	}
	\label{fig:AreaAgg_CaseStudy1_Rg543}
\end{figure}


\begin{figure}[tb]
	\centering
	\includegraphics[page=3]{AreaAgg_CaseStudy1}
	\caption{An optimal sequence of intermediate subdivisions 
		of region~$53$ obtained by \Astar 
		using costs of type change and compactness.		 
		The numbers indicate the numbers of patches.		
	}
	\label{fig:AreaAgg_CaseStudy1_Rg53}
\end{figure}


%\todo[inline]{It turned out that the objective function $g$ 
%seems rather robust
%	against changes of~$\lambda$.  For region~77, we 
%	varied~$\lambda$ in
%	the range $\{0.1,0.5,0.9\}$, but the resulting aggregation 
%	sequences
%	stayed almost the same.  When we used even larger 
%	values 
%	for~$\lambda$ (e.g.,~$0.99$), the (good) influence of type 
%	change was mitigated 
%	and, as a
%	result, we needed to overestimate (with factor~$K=8$).}




\subsection{Using costs of type change and length}
\label{sec:AreaAgg_CaseStudy2}

We compare the greedy algorithm, \Astar, and our ILP
using~$g_2(\Pnode)$,
which is a combination of the costs 
of type change and length (see \eq\ref{eq:g_2}).
For \Astar, we overestimated 
whenever we could not find a solution after 
having visited~$W=200{,}000$ nodes (see 
Sec.~\ref{sec:AreaAgg_AStar}).
The most time-consuming instance for \Astar was region~$94$,
for which \Astar took~$160.0\,$s to find a feasible solution 
with overestimation factor~$K=31$.
To avoid waiting too long,
we set the time limit as~$160\,$s 
for our ILP to solve the optimization problem on one region.
Note that the time limit does not take into account
the time that our ILP uses 
to set up the variables and constraints
(see \sects\ref{sub:AreaAgg_variables} 
and~\ref{sub:AreaAgg_Constraints}).
If no optimal solution is found in this amount of time,
a feasible solution (if found) will be returned.
For some large instances, 
the ILP cannot find any solution.

Using a similar format as in \tab\ref{tab:AreaAgg_CaseStudy1_Statistics},
we present the statistics in 
\tab\ref{tab:AreaAgg_CaseStudy2_Statistics}.
\Astar found optimal solutions for~$695$ of the~$734$ regions.
Again, it spent most of the running time 
on the few regions that needed 
overestimation:~$5.3\%$ of the regions 
cost~$89.5\%$ of the total running time.
The solutions by \Astar cost~$438.2$ in total, which is~$3.9\%$ 
less than~$455.8$, the total cost of the solutions by the greedy 
algorithm.

\begin{table*}[tb]
	%\small	
	\centering
	\caption{A comparison of the greedy algorithm and \Astar		
		when using cost function~$g_2$ (see \eq\ref{eq:g_2}).
		The notations are the same as in
		\tab\ref{tab:AreaAgg_CaseStudy1_Statistics}.
		%
		$\sum g_\mathrm{type}$, $\sum g_\mathrm{comp}$, 
		and~$\sum g_1$
		respectively denotes the sums of~$g_\mathrm{type}(\Pgoal)$,
		$g_\mathrm{lgth}(\Pgoal)$, and~$g_2(\Pgoal)$ 
		over all the~$734$ instances 
		(see \eqs\ref{eq:g_type}, \ref{eq:g_length}, 
		and~\ref{eq:g_2}).
	}
	\label{tab:AreaAgg_CaseStudy2_Statistics}
	%	\includegraphics[page=1]{AreaAgg_CaseStudy2_Plot}
	\setlength{\tabcolsep}{0.7ex}
	\begin{tabular}{lCRCCCCCd{3.7}}
		\toprule
		\multicolumn{1}{l}{Methods} &
		\multicolumn{1}{c}{\#over} &
		\multicolumn{1}{r}{$k_\mathrm{sum}$} &  
		\multicolumn{1}{c}{\#nodes} & 
		\multicolumn{1}{c}{\#arcs} & 
		\multicolumn{1}{c}{$\sum g_\mathrm{type}$} & 
		\multicolumn{1}{c}{$\sum g_\mathrm{comp}$} & 
		\multicolumn{1}{c}{$\sum g_1$} & 
		\multicolumn{1}{c}{Time (min)} \\ 
		\midrule
		Greedy 	&     &     & 5.5\cdot 10^3 & 4.8\cdot 10^3 & 
		53.0 & 
		858.5 & 455.8 & 2'0\\
		%
		\Astar$_{\!\!200{,}000}$	&  39~(5.3\%) & 150 &  3.7 
		\cdot 
		10^6 &    7.3 \cdot 10^6 & 52.0 & 824.5 & 438.2 & 
		62'0~(89.5\%)
		%
		\\ \bottomrule			
	\end{tabular}
\end{table*}




\Astar managed to find optimal solutions 
for all the regions with fewer than~$15$ polygons, and 
found only feasible solutions for the regions 
with more than~$21$ polygons.
%
In the~$39$ (out of~$734$) regions 
that \Astar failed to solve optimally,
the greedy algorithm outperformed \Astar 
for~$25$ regions~($64.1\%$), 
which is~$17.2\%$ more comparing to the first experiment.
%
The ILP managed to find optimal solutions 
for all the regions with fewer than~$8$ polygons,
and failed to find optimal solutions
for any region with more than~$8$ polygons.
In none of the~$39$ regions that \Astar failed to solve 
optimally
did the ILP find a feasible solution.
Overall, the ILP found optimal solutions for~$442$ regions,
and found feasible solutions for~$72$ regions.
There are~$220$ regions for which 
our ILP failed to find any solution.
For~$17$ of those regions, the main memory was not enough to 
set up the variables and the constraints.
For other~$44$ of those regions, 
our ILP ran out of the main memory 
before finding any feasible solution.
Note that we allowed our program to use~$3\,$GB 
of the main memory at most.
After we increased the time limit to~$600\,$s for each region, 
the ILP solved~$491$ regions to optimality, which 
is~$49$ regions more than with the time limit of~$160\,$s.
Each of the~$49$ regions has~$6$ to~$10$ polygons.
\fig\ref{fig:AreaAgg_CaseStudy2_Percentage_Optimal} 
shows the percentages of the regions 
that are solved optimally by the ILP with the two time limits.
\fig\ref{fig:AreaAgg_CaseStudy2_ILP} shows 
the number of regions for which 
the ILP found optimal, feasible, or no solutions 
when using the two time limits~$160\,$s and~$600\,$s.


\begin{figure}[tb]
	\centering
	\includegraphics[page=1]{AreaAgg_CaseStudy2_Plot}
	\caption{The percentage of regions that can be solved 
		optimally by \Astar or our ILP.
		Note that the numbers of regions according to~$n$ 
		(the number of polygons on the start map in one region) 
		are shown in \fig\ref{fig:AreaAgg_NumRegion}.}
	\label{fig:AreaAgg_CaseStudy2_Percentage_Optimal}
	%
	\par\vspace{\baselineskip} %Leave a gap	between figures
	%
	\centering
	\includegraphics[page=2]{AreaAgg_CaseStudy2_Plot}
	\caption{The percentage of regions for which we found at 
		least feasible solutions by \Astar or our ILP.
		Note that the numbers of regions according to~$n$ 
		(the number of polygons on the start map in one region) 
		are shown in \fig\ref{fig:AreaAgg_NumRegion}.}
	\label{fig:AreaAgg_CaseStudy2_Percentage_Feasible}
\end{figure}

\begin{figure}[tb]
	\centering
	\includegraphics[page=3]{AreaAgg_CaseStudy2_Plot}
	\caption{The number of regions for which
		our ILP found optimal, feasible, or no solutions 
		when using time limits $160\,$s and $600\,$s.
		Using more time, our ILP was able to 
		solve more instances to optimality.}
	\label{fig:AreaAgg_CaseStudy2_ILP}
\end{figure}

Among all the instances that were solved to optimality by \Astar
in both experiments (i.e., \sects\ref{sec:AreaAgg_CaseStudy1} 
and~\ref{sec:AreaAgg_CaseStudy2}),
region~$358$
(marked in \tab\ref{tab:CostsInDetail})
is the largest one.
In both experiments, the cost of type change is~$0.044$.
The optimal aggregation sequences for this region
obtained by using costs~$g_1$ and~$g_2$
are shown in \fig\ref{fig:AreaAgg_CaseStudy2_Rg358}.
We, however, noticed some unpleasant aggregates.
The step from~$8$ patches to~$7$ patches 
when using cost function~$g_1$ is a bad move
(see \fig\ref{fig:AreaAgg_CaseStudy2_Rg358}b).
Instead we expect the result of 
\fig\ref{fig:AreaAgg_CaseStudy2_Rg358}a.
Using cost function~$g_2$, we had a similar problem. 
The subdivision with~$7$ patches is such an example,
where we expect the result of 
\fig\ref{fig:AreaAgg_CaseStudy2_Rg358}c.
In an earlier version of this paper \parencite{Peng2017AStar},
we tried minimizing type changes 
and maximizing the sum of the smallest compactness values, 
over the whole sequence.
For that objective, we had a similar problem as 
in \fig\ref{fig:AreaAgg_CaseStudy2_Rg358}b.
This problem, however, can be fixed easily 
by forbidding two patches to aggregate 
if their common boundary is too short.


\begin{figure}[h!tb]
	\centering
	\includegraphics[page=1]{AreaAgg_CaseStudy2}
	\caption{Some intermediate subdivisions of region~$358$ 
		obtained by \Astar with different cost functions.
		The numbers indicate the numbers of patches.
	}
	\label{fig:AreaAgg_CaseStudy2_Rg358}
\end{figure}


%\todo{chart: memory use of greedy, A*, and ILP}

\section{Concluding Remarks}
\label{sec:AreaAgg_Conclusions}
In this paper, we investigated the problem of 
finding optimal sequences for area aggregation.
We compared three methods to solve this problem, namely, 
a greedy algorithm, \Astar, and an ILP-based algorithm.
The greedy algorithm is used as a benchmark.
Unsurprisingly, it ran faster than the other two methods by far.
According to our experiments, \Astar found area aggregation 
sequences
with the least total cost over all regions.
For some instances, however, \Astar had to overestimate
in order to find feasible solutions.
Compared to the greedy algorithm, 
\Astar reduced the total cost by~$3.9\%$.
Although the amount is small, it is worth to use \Astar
because optimization methods can help us to evaluate the quality 
of a model \parencite{Haunert2017Label,Haunert2008Assuring,Haunert2016Optimization}.
For example, \fig\ref{fig:AreaAgg_CaseStudy2_Rg358}b shows that
even an \emph{optimal} sequence can be problematic. 
This result makes us realize that our model of minimizing the 
type change and the compactness has drawbacks.
The ILP-based algorithm finds optimal solutions for some regions,
but for some of the other regions 
it cannot even find a feasible solution.
Compared to the ILP-based algorithm,
\Astar used less memory 
yet found optimal solutions for more regions.

%\todo[inline]{estimate compactness based on length; cauchy 
%schwarz}

For \Astar, we have a good estimation for the cost of type 
change, which helps a lot to reduce the search space. 
Our estimation for the cost of shape 
(compactness or length) is poor.
There are two ways to improve \Astar
in terms of solving more instances to optimality 
while using the same limit of main memory.
First, during the searching we can forget a node of the graph
if all the neighbors of this node have been visited.
By testing a case, we learned that half of the nodes can be 
forgotten during the pathfinding process.
In this way, we can release some main memory 
and visit more nodes.
Once we arrive at the goal, 
we know the cost for 
an optimal solution (the least cost).
As many visited nodes have been forgotten, 
we do not have the shortest path so far.
We need to run \Astar again.
This time we know for sure that 
a path is not optimal if its cost, 
the sum of the exact cost and the estimated cost, 
is more than the least cost 
(of the optimal solution found previously).
Consequently, we are able to prune some branches earlier 
than the first time we run \Astar.
In this way, we manage to save some main memory.
As a result, we are more likely to find optimal solutions
when the main memory is limited.
Second, if we obtain a solution based on overestimation, 
then we know the cost of this non-optimal solution.
We may decrease the overestimation factor by pruning the branches
that cost more than the non-optimal solution.


We may speed up our ILP-based algorithm using a so-called 
cutting-plane approach as \textcite{Oehrlein2017Aggregation}.
Also, we can add more constraints to
reduce the choices of variables.
For example, assignment to a given center~$r$ is symmetric, 
hence
\begin{equation*}
\label{eq:CstrZX}
z_{t,p,q,r}= z_{t,p,q,r} \qquad
\forall t \in {T} \setminus \{1,n\}, 
\forall p, q, r \in P.
\end{equation*}
Whether adding such kinds of constraints always
speeds up our ILP is not clear
because the solver, CPLEX, is a black box to us.
Although integer linear programming may be not good at 
finding optimal sequences for area aggregation,
it is relatively easy to formulate problems as ILPs.
As stated by \citet[p.~861]{Cormen2009}, 
``an efficient algorithm designed specifically for a problem 
will often be more efficient than 
linear programming both in theory and in practice. 
The real power of linear programming comes from 
the ability to solve new problems.''

We may improve both the \Astar algorithm and the ILP-based 
algorithm by integrating the greedy algorithm.
The idea is that we use the greedy algorithm to find 
a solution. 
Then we can use the cost of the solution as an upper bound to 
prune the branches of \Astar and the ILP. 
Once we see that
the cost of a branch is larger than the upper bound,
we can ignore this branch  
because it will not yield an optimal solution.



In cartography, there are many more requirements 
for area aggregation.
For example, one requirement is 
to keep important land-cover areas for a longer time
(such as a settlement surrounded by farmlands).
This requirement can be achieved by incorporating the idea of
\textcite{Dilo2009tGAP}.
They gave each type a weight, then defined the importance of a patch by the product of the area size and the type weight.
While in our method, we used only the area size as importance.
%
Another requirement is that 
aggregating two areas may result in 
an area with a generalized type,
as did in \textcite{vanSmaalen2003}. 
For example, aggregating 
farmland with hedge yields an area with type vegetation.
%
These issues can be considered in our future work.
%




\begin{acks}
We thank Thomas C. van Dijk, Joachim Spoerhase, 
and Sabine Storandt for their valuable suggestions.
We thank Martijn Meijers for 
prereviewing an earlier version of this paper.
\end{acks}


\printbibliography


\end{document}
